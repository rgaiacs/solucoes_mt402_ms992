% Filename: lista14.tex
% 
% This code is part of 'Solutions for MT402, Matrizes'
% 
% Description: This file corresponds to the solutions of homework sheet 14.
% 
% Created: 08.05.12 09:34:01 PM
% Last Change: 29.06.12 05:40:35 PM
% 
% Authors:
% - Raniere Silva (2012): initial version
% 
% Copyright (c) 2012 Raniere Silva <r.gaia.cs@gmail.com>
% 
% This work is licensed under the Creative Commons Attribution-ShareAlike 3.0 Unported License. To view a copy of this license, visit http://creativecommons.org/licenses/by-sa/3.0/ or send a letter to Creative Commons, 444 Castro Street, Suite 900, Mountain View, California, 94041, USA.
%
% This work is distributed in the hope that it will be useful, but WITHOUT ANY WARRANTY; without even the implied warranty of MERCHANTABILITY or FITNESS FOR A PARTICULAR PURPOSE.
%
Considere a matriz $A \in \mathbb{C}^{n \times n}$ e $\lambda_i$ seus autovalores.
\begin{questions}
    \question A matriz $A$ \'{e} singular se e somente se $\lambda = 0$ \'{e} um autovalor de $A$.
    \begin{solution}
        Primeiro vamos mostrar que se $A$ \'{e} singular ent\~{a}o $\lambda = 0$ \'{e} um autovalor.

        Se $A$ \'{e} singular ent\~{a}o $\det(A)=0$ e, portanto, $\det(A - 0I) = 0$. Logo, $0$ \'{e} autovalor.

        Agora vamos mostrar que se $\lambda = 0$ \'{e} um autovalor ent\~{a}o $A$ \'{e} singular.

        Se $\lambda = 0$ \'{e} um autovalor ent\~{a}o $\det(A - 0I) = 0$ e, portanto $\det(A) = 0$. Logo, $A$ \'{e} singular.
    \end{solution}

    \question Se $A$ \'{e} n\~{a}o singular e se $(\lambda, v)$ \'{e} um autopar de $A$, ent\~{a}o, $(\lambda^{-1}, v)$ \'{e} um autopar de $A^{-1}$.
    \begin{solution}
        Se $A$ \'{e} n\~{a}o singular ent\~{a}o existe $A^{-1}$ tal que $A A^{-1} = A^{-1} A = I$. Se $(\lambda, v)$ \'{e} autopar de $A$ ent\~{a}o
        \begin{align*}
            A v &= \lambda v \\
            A^{-1} A v &= A^{-1} \lambda v \\
            I v &= \lambda A^{-1} v \\
            \lambda^{-1} v &= A^{-1} v,
        \end{align*}
        i.e., $(\lambda^{-1}, v)$ \'{e} um autopar de $A^{-1}$.
    \end{solution}

    \question $A$ e $A^t$ possuem os mesmos autovalores? Justifique.
    \begin{solution}
        $A$ e $A^t$ possuem os mesmos autovalores. Suponhamos que $\lambda$ \'{e} um autovalor de $A$, ent\~{a}o $\det(A - \lambda I) = 0$ e portanto
        \begin{align*}
            \det(A - \lambda I)^t &= \det\left( (A - \lambda I)^t \right) \\
            &= \det(A^t - (\lambda I)^t) \\
            &= \det(A^t - \lambda I).
        \end{align*}
        Logo, $\lambda$ \'{e} um autovalor de $A^t$.
    \end{solution}

    \question $A$ e $A^H$ possuem os mesmos autovalores? Justifique.
    \begin{solution}
        $A$ e $A^H$ n\~{a}o possuem os mesmos autovalores. Suponhamos que $\lambda$ \'{e} um autovalor de $A$, ent\~{a}o $\det(A - \lambda I) = 0$ e portanto
        \begin{align*}
            \det(A - \lambda I)^H &= \det\left( (A - \lambda I)^H \right) \\
            &= \det(A^H - \lambda^H I) \\
            &= \det(A^H - \bar{\lambda} I).
        \end{align*}
        Logo, $\bar{\lambda}$ \'{e} um autovalor de $A^H$.
    \end{solution}

    \question A matriz $\alpha A$ possui autovalores $\alpha \lambda_i$.
    \begin{solution}
        Seja $(\lambda, v)$ um autopar de $A$, ent\~{a}o $A v = \lambda v$ e
        \begin{align*}
            (\alpha A) v &= \alpha A v \\
            &= \alpha \lambda v.
        \end{align*}
        Logo, $\alpha \lambda$ \'{e} um autovalor de $\alpha A$.
    \end{solution}

    \question Os autovalores de $A^r$ s\~{a}o $\lambda_i^r$, onde $r$ \'{e} um inteiro positivo.
    \begin{solution}
        Seja $(\lambda, v)$ um autopar de $A$, ent\~{a}o $A v = \lambda v$ e
        \begin{align*}
            A^2 v &= A (A v) \\
            &= A A v \\
            &= A \lambda v \\
            &= \lambda^2 v.
        \end{align*}
        Aplicando repetidamente o processo anterior concluimos que $\lambda^r$ \'{e} um autovalor de $A^r$, onde $r$ \'{e} um inteiro positivo.
    \end{solution}

    \question Se $A$ \'{e} hermitiana ent\~{a}o seus autovalores s\~{a}o reais.
    \begin{solution}
        Suponhamos que $\lambda$ \'{e} um autovalor de $A$, ent\~{a}o $\det(A - \lambda I) = 0$ e portanto
        \begin{align*}
            \det(A - \lambda I)^H &= \det\left( (A - \lambda I)^H \right) \\
            &= \det(A^H - \lambda^H I) \\
            &= \det(A - \overline{\lambda} I).
        \end{align*}
        Logo, $\lambda = \overline{\lambda}$, i.e., $\lambda \in \mathbb{R}$, \'{e} um autovalor de $A$.
    \end{solution}

    \question Se $A$ \'{e} real sim\'{e}trica ent\~{a}o seus autovalores s\~{a}o reais.
    \begin{solution}
        Seja $(\lambda, v)$ um autopar da matriz $A$, ent\~{a}o $A v = \lambda v$ e $\overline{A v} = \overline{\lambda v}$. Como $A$ \'{e} real, temos que $A \overline{v} = \overline{\lambda v}$. Ent\~{a}o
        \begin{align*}
            \lambda (v \overline{v}) &= (\lambda v) \overline{v} = A v \overline{v} = A \overline{v} v = \overline{\lambda v} v = \overline{\lambda} (v \overline{v})
        \end{align*}
        e portanto $\lambda = \overline{\lambda}$ que implica em $\lambda \in \mathbb{R}$.
    \end{solution}

    \question Se $A$ \'{e} real sim\'{e}trica definida positiva ent\~{a}o seus autovalores s\~{a}o positivos.
    \begin{solution}
        Seja $(\lambda, v)$ um autopar da matriz $A$, ent\~{a}o $A v = \lambda v$. Ent\~{a}o
        \begin{align*}
            v^t A v &= v^t \lambda v = \lambda \| v \|.
        \end{align*}
        Como $A$ \'{e} sim\'{e}trica definida positiva temos que $v^t A v > 0$, $v \neq 0$ e portanto $\lambda > 0$.
    \end{solution}

    \question Se $A$ \'{e} ortogonal ent\~{a}o $| \lambda_i | = 1$.
    \begin{solution}
        Se $A$ \'{e} ortogonal ent\~{a}o $A^H A = I$. Seja $(\lambda, v)$ um autopar de $A$ e portanto,
        \begin{align*}
            A v &= \lambda v \\
            A^H A v &= A^H \lambda v \\
            I v &= \lambda A^H v \\
            v &= \lambda \bar{\lambda} v.
        \end{align*}
        Logo, $\lambda \bar{\lambda} = 1$ que corresponde a $|\lambda| = 1$.
    \end{solution}

    \question Para as normas induzidas a partir das normas vetoriais $p$: $\| A \|_p \geq | \lambda_i |, \forall i$.
    \begin{solution}
        Seja $(\lambda, v)$ um autopar de $A$, ent\~{a}o
        \begin{align*}
            \| A \|_p &= \max \| A x \|_p / \| x \| p \\
            &\geq \max \| A v \|_p / \| v \|_p \\
            &= \max \| \lambda v \|_p / \| v /|_p \\
            &\geq max | \lambda | \| v \|_p / \| v \|_p \\
            &= \lambda.
        \end{align*}
    \end{solution}

    \question Os autovalores de $A + \alpha I$ s\~{a}o $\lambda_i + \alpha$.
    \begin{solution}
        Seja $(\lambda, v)$ um autopar de $A$, ent\~{a}o $A v = \lambda v$ e
        \begin{align*}
            (A + \alpha I) v &= A v + \alpha I v \\
            &= \lambda v + \alpha v \\
            &= (\lambda + \alpha)v.
        \end{align*}
        Logo, $\lambda + \alpha$ \'{e} autovalor de $A + \alpha I$.
    \end{solution}

    \question Enuncie e demonstre o teorema dos discos de Gerchgorin. Exemplifique como empregar o resultado deste teorema para localiza\c{c}\~{a}o dos autovalores de uma matriz $A$.
    \begin{solution}
        % TODO Fazer esse exerc\'{i}cio.
    \end{solution}

    \question Usando o teorema dos discos de Gerschgorin, demonstre que uma matriz diagonalmente dominante \'{e} n\~{a}o singular.
    \begin{solution}
        % TODO Fazer esse exerc\'{i}cio.
    \end{solution}

    \question Considerando o teorema abaixo
    \begin{quote}
        Se a uni\~{a}o $\mathcal{U}$ de $k$ discos de Gerschgorin n\~{a}o tangencia qualquer um dos demais $n - k$ c\'{i}rculos, ent\~{a}o existem exatamente $k$ autovalores, contando suas multiplicidades, nos discos em $\mathcal{U}$. (Meyer\nocite{Meyer:2000:matrix}, cap. 7, p\'{a}g, 498).
    \end{quote}
    verifique que a matriz
    \begin{align*}
        A &= \begin{bmatrix}
            1 & 0 & -2 & 0 \\
            0 & 12 & 0 & -4 \\
            1 & 0 & -1 & 0 \\
            0 & 5 & 0 & 0
        \end{bmatrix}
    \end{align*}
    possui pelo menos dois autovalores reais.
    \begin{solution}
        % TODO Fazer esse exerc\'{i}cio.
    \end{solution}

    \question Demonstre que a matriz $A : n \times n$, $a_{ii} = n$, $i = 1, \ldots, n$ e $a_{ij} = 1$, $i \neq j$ \'{e} n\~{a}o singular.
    \begin{solution}
        % TODO Fazer esse exerc\'{i}cio.
    \end{solution}
\end{questions}
