% Filename: lista09.tex
% 
% This code is part of 'Solutions for MT402, Matrizes'
% 
% Description: This file corresponds to the solutions of homework sheet 09.
% 
% Created: 23.04.12 07:00:39 PM
% Last Change: 04.06.12 05:12:20 PM
% 
% Authors:
% - Raniere Silva (2012): initial version
% 
% Copyright (c) 2012 Raniere Silva <r.gaia.cs@gmail.com>
% 
% This work is licensed under the Creative Commons Attribution-ShareAlike 3.0 Unported License. To view a copy of this license, visit http://creativecommons.org/licenses/by-sa/3.0/ or send a letter to Creative Commons, 444 Castro Street, Suite 900, Mountain View, California, 94041, USA.
%
% This work is distributed in the hope that it will be useful, but WITHOUT ANY WARRANTY; without even the implied warranty of MERCHANTABILITY or FITNESS FOR A PARTICULAR PURPOSE.
%
\documentclass[a4paper,12pt, leqno, answers]{exam}
% Customiza\c{c}\~{a}o da classe exam
\newcommand{\mycheader}{Lista 09 - Fatora\c{c}\~{o}es Ortogonais}
\header{MT402}{\mycheader}{\thepage/\numpages}
\headrule
\footer{Dispon\'{i}vel em \\\url{https://github.com/r-gaia-cs/solucoes_lista_matrizes}
}{}{Reportar erros para \\% Filename: maintainer.tex
% 
% This code is part of 'Solutions for MT402, Matrizes'
% 
% Description: This file keeps the email of the mainteiner.
% 
% Created: 08.05.12 09:46:55 PM
% Last Change: 04.06.12 10:42:01 PM
% 
% Authors:
% - Raniere Silva (2012): initial version
% 
% Copyright (c) 2012 Raniere Silva <r.gaia.cs@gmail.com>
% 
% This work is licensed under the Creative Commons Attribution-ShareAlike 3.0 Unported License. To view a copy of this license, visit http://creativecommons.org/licenses/by-sa/3.0/ or send a letter to Creative Commons, 444 Castro Street, Suite 900, Mountain View, California, 94041, USA.
%
% This work is distributed in the hope that it will be useful, but WITHOUT ANY WARRANTY; without even the implied warranty of MERCHANTABILITY or FITNESS FOR A PARTICULAR PURPOSE.
%
\href{mailto:r.gaia.cs@gmail.com}{r.gaia.cs@gmail.com}
}
\footrule 
\pagestyle{headandfoot}
\renewcommand{\solutiontitle}{\noindent\textbf{Solu\c{c}\~{a}o:}\enspace}
\SolutionEmphasis{\slshape}
\unframedsolutions
\pointname{}

% Filename: paper_size.tex
% 
% This code is part of 'Solutions for MT402, Matrizes'
% 
% Description: This file corresponds to the paper size output.
% 
% Created: 08.05.12 09:46:55 PM
% Last Change: 04.06.12 10:42:01 PM
% 
% Authors:
% - Raniere Silva (2012): initial version
% 
% Copyright (c) 2012 Raniere Silva <r.gaia.cs@gmail.com>
% 
% This work is licensed under the Creative Commons Attribution-ShareAlike 3.0 Unported License. To view a copy of this license, visit http://creativecommons.org/licenses/by-sa/3.0/ or send a letter to Creative Commons, 444 Castro Street, Suite 900, Mountain View, California, 94041, USA.
%
% This work is distributed in the hope that it will be useful, but WITHOUT ANY WARRANTY; without even the implied warranty of MERCHANTABILITY or FITNESS FOR A PARTICULAR PURPOSE.
%
% Para impress\~{a}o
\usepackage[top=3cm, bottom=3cm, left=2cm, right=2cm]{geometry}

% Para ereaders (Kindle, Nook, Kobo, ...) and tablets (iPad, GalaxyTab, ...)
% \usepackage[papersize={180mm,240mm},margin=2mm]{geometry}
% \sloppy


% Pacotes
\usepackage[utf8]{inputenc}
\usepackage[T1]{fontenc}
\usepackage[brazil]{babel}
\usepackage{amsmath}
\usepackage{amsfonts}
\usepackage{amssymb}
\usepackage{hyperref}
\usepackage{graphicx}
\usepackage{tikz}
\usepackage{algorithmicx}
\usepackage{algpseudocode}

% Customiza\c{c}\~{a}o do pacote amsmath
\allowdisplaybreaks[4]

% Novos ambientes
% \newenvironment{fwsolution}{\begin{EnvFullwidth}\begin{TheSolution}}{\end{TheSolution}\end{EnvFullwidth}}

% Novos comandos
% \newcommand{\mdot}{\text{\LARGE $\boldsymbol{\cdot}$}}
\newcommand{\mdot}{\bullet}

\begin{document}
%cover
\thispagestyle{empty}
\input{cover.tex}
\newpage
\setcounter{page}{1}
\begin{questions}
    \question Seja $w \in \mathbb{R}^n$ tal que $\| w \|_2 = 1$ e defina $Q : n \times n$ por $Q = I - 2 w w^t = I - 2 P$ onde $P = w w^t$. Demonstre que
    \begin{parts}
        \part $Q w = - w$;
        \begin{solution}
            \begin{align*}
                Q w &= \left( I - 2 w w^t \right) w \\
                &= w - 2 w w^t w \\
                &= w - 2 w && w^t w = \| w \|_2 = 1 \\
                &= -w.
            \end{align*}
        \end{solution}

        \part $Q v = v$ se $<w, v> = 0$;
        \begin{solution}
            \begin{align*}
                Q v &= \left( I - 2 w w^t \right) v \\
                &= v - 2 w w^t v \\
                &= v - \left( 2 w \right) 0 && w^t v = <w, v> = 0 \\
                &= v.
            \end{align*}
        \end{solution}

        \part $Q$ \'{e} sim\'{e}trica e ortogonal.
        \begin{solution}
            Primeiro vamos mostrar que $Q$ \'{e} sim\'{e}trica, i.e., $Q = Q^t$.
            \begin{align*}
                Q^t &= (I - 2 P)^t \\
                &= I^t - 2 P^t \\
                &= I - 2 P = Q && \text{$P$ \'{e} sim\'{e}trica.}
            \end{align*}
        \end{solution}
    \end{parts}

    \question Dado $z = (5, 2)^t$. Defina: $v = \left( -\| z \|_2, 0 \right)^t$ e $u = z - v$. Interprete geometricamente a a\c{c}\~{a}o das matrizes $P = \left( u u^t \right) / u^t u$, $R = I - P$ e $Q = I - 2P$ em vetores do $\mathbb{R}^2$.
    \begin{solution}
        Pela defini\c{c}\~{a}o apresentada:
        \begin{align*}
            v &= \left( -\sqrt{29}, 0 \right)^t \sim \left( -5.39, 0 \right)^t, \\
            u &= \left( 5 + \sqrt{29}, 2 \right)^t \sim \left( 10.39, 0 \right)^t, \\
            P &= \frac{1}{2 \left( 29 + \sqrt{29} \right)} \begin{bmatrix}
                \left( 5 + \sqrt{29} \right)^2 & 2 \left( 5 + \sqrt{29} \right) \\
                2 \left( 5 + \sqrt{29} \right) & 2^2
            \end{bmatrix} \sim \begin{bmatrix}
                0.96 & 0.19 \\
                0.19 & 0.04
            \end{bmatrix}, \\
            R &= \begin{bmatrix}
                1 - \left( 5 + \sqrt{29} \right)^2 / \left[ 2 \left( 29 + \sqrt{29} \right) \right] & \left( 5 + \sqrt{29} \right) / \left( 29 + \sqrt{29} \right) \\
                \left( 5 + \sqrt{29} \right) / \left( 29 + \sqrt{29} \right) & 1 - 2^2 / \left[ 2 \left( 29 + \sqrt{29} \right) \right]
            \end{bmatrix} \sim \begin{bmatrix}
                0.04 & -0.19 \\
                -0.19 & 0.96
            \end{bmatrix}, \\
            Q &= \begin{bmatrix}
                1 - 2 \left( 5 + \sqrt{29} \right)^2 / \left[ 2 \left( 29 + \sqrt{29} \right) \right] & - 2 \left( 5 + \sqrt{29} \right) / \left( 29 + \sqrt{29} \right) \\
                - 2 \left( 5 + \sqrt{29} \right) / \left( 29 + \sqrt{29} \right) & 1 - 2 \cdot 2^2 / \left[ 2 \left( 29 + \sqrt{29} \right) \right]
            \end{bmatrix} \sim \begin{bmatrix}
                -0.93 & -0.37 \\
                -0.37 & 0.93
            \end{bmatrix}.
        \end{align*}

        As matrizes $P$, $R$ e $Q$ correspondem, respectivamente, a um projetor, um projetor orgotonal e um refletor. Geometricamente temos
        \begin{center}
            \begin{tikzpicture}
                \draw[->] (-6, 0) -- (10, 0) node[below right] {$x$};
                \draw[->] (0, -1) -- (0, 3) node[above right] {$y$};

                \draw[->, blue] (0,0) -- (5,2) node[below right, black] {$z$};
                \draw[->, blue] (0,0) -- (sqrt{29}, 0) node[below right, black] {$v$};
                \draw[->, blue] (0,0) -- (5 + sqrt{29}, 2) node[below right, black] {$u$};
                \draw[->, blue] (0,0) -- (5.1926, 1.0000) node[below right, black] {$P z$};
                \draw[->, blue] (0,0) -- (-0.19258, 1.00000) node[below right, black] {$R z$};
                \draw[->, blue] (0,0) -- (-5.38516, 0.00000) node[below right, black] {$Q z$};
            \end{tikzpicture}
        \end{center}
    \end{solution}

    \question Considere $u : n \times 1$, $\| u \|_2 = 1$ e $P = u u^t$. Justificando todos os argumentos usados:
    \begin{parts}
        \part fa\c{c}a uma interpreta\c{c}\~{a}o geom\'{e}trica da a\c{c}\~{a}o das matrizes $P$, $I - P$ e $I - 2P$ em $v \in \mathbb{R}^n$;
        \begin{solution}
            As matrizes $P$, $I - P$ e $I - 2P$ correspondem, respectivamente, a um projetor, um projetor ortogonal e um refletor.
            
            Em maiores detalhes temos:
            \begin{itemize}
                \item $P v$ corresponde a proje\c{c}\~{a}o de $v$ em $u$;
                \item $\left( I - P \right) v$ corresponde a proje\c{c}\~{a}o de $v$ no plano perpendicular a $u$;
                \item $\left( I - 2 P \right) v$ corresponde a reflex\~{a}o de $v$ no plano perpendicular a $u$.
            \end{itemize}
        \end{solution}

        \part o que podemos afirmar a respeito do posto de cada uma das matrizes do item anterior?
        \begin{solution}
            O posto da matriz $P$ \'{e} 1 pois as linhas s\~{a}o múltiplos do vetor $u$. O posto da matriz $I - P$ \'{e} $n - 1$ pois $\text{Im}(I - P) = \text{N}(P)$. E o posto de $I - 2 P$ \'{e} $n$ pois a matriz \'{e} n\~{a}o singular.
        \end{solution}
    \end{parts}

    \question Demonstre que se $P$ \'{e} um projetor ortogonal ent\~{a}o $I - 2 P$ \'{e} uma matriz sim\'{e}trica e ortogonal. Interprete geometricamente.
    \begin{solution}
        Se $P$ \'{e} um projetor ortogonal ent\~{a}o $P^2 = P$ e $P^t = P$.

        Primeiro vamos mostrar que $I - 2 P$ \'{e} sim\'{e}trica.
        \begin{align*}
            \left( I - 2 P \right)^t &= I^t - 2 P^t \\
            &= I - 2 P && P^t = P.
        \end{align*}

        Agora vamos mostrar que $I - 2 P$ \'{e} ortogonal.
        \begin{align*}
            \left( I - 2 P \right)^t \left( I - 2 P \right) &= \left( I - 2 P \right) \left( I - 2 P \right) \\
            &= I - 2 P - 2 P + 4 P P \\
            &= I - 4 P + 4 P && P P = P \\
            &= I.
        \end{align*}
    \end{solution}
    
    \question Fatora\c{c}\~{a}o QR reduzida

    Se $A : m \times n$, $\text{posto}(A) = n$. O processo de Gram Schmidt pode ser aplicado para construir as matrizes $\hat{Q} : m \times n$ and $\hat{R} : n \times n$, $r_{ij} = 0$, $i > j$, denominada fatora\c{c}\~{a}o ortogonal reduzida de $A$. 
    
    Esta fatora\c{c}\~{a}o \'{e} deduzida, impondo que dados os vetores colunas de $A : \left\{ a_1, a_2, \ldots, a_n \right\}$ queremos obter os vetores $\left\{ q_1, q_2, \ldots, q_n \right\}$ de modo que estes vetores sejam ortogonais e gerem o mesmo subespa\c{c}o, isto \'{e}, $\text{Im}(A) = \text{Im}(\hat{Q})$. Esta condi\c{c}\~{a}o \'{e} satisfeita se escrevermos $A$ como o produto de $\hat{Q}$ por uma outra matriz, que denonominaremos $\hat{R}$. ($A = \hat{Q} \hat{R}$: colunas de $A$ s\~{a}o combina\c{c}\~{o}es lineares das colunas de $R$ e as linhas de $A$ s\~{a}o combina\c{c}\~{o}es lineares das linhas de $\hat{R}$). Al\'{e}m disto, podemos impor que $\hat{R}$ seja triangular superior. Da equa\c{c}\~{a}o matricial: $A = \hat{Q} \hat{R}$, e do fato que as colunas de $\hat{Q}$ devem ser ortogonais e $\hat{R}$ triangular superior, deduzimos as f\'{o}rmulas para os vetores $q_j$ e elementos $r_{ij}$, que \'{e} o processo de Gram Schmidt. Verifique que as colunas $q_j$ de $\hat{Q}$ e os elementos $r_{ij}$ de $\hat{R}$ s\~{a}o dados por:
    \begin{algorithmic}
        \For{$j = 1, \ldots, n$}
            \For{$i = 1, \ldots, j - 1$}
                \State $r_{ij} \leftarrow q_i^t a_j$
            \EndFor
            \State $v_j \leftarrow a_j - \sum_{i = 1}^{j - 1} r_{ij} q_i$
            \State $r_{jj} \leftarrow \| v_j \|_2$
            \State $q_j \leftarrow v_j / r_{jj}$
        \EndFor
    \end{algorithmic}
    \begin{solution}
        N\'{a} p\'{a}gina 231 do Golub\nocite{Golub:1996:matrix} encontramos o algoritmo de Gram-Schmidt que \'{e} transcrito a seguir.
    \begin{algorithmic}
        \State $R(1, 1) \leftarrow \| A(:, 1) \|_2$
        \State $Q(:, 1) \leftarrow A(:, 1) / R(1, 1)$
        \For{$k = 2, \ldots, n$}
            \State $R(1:k - 1, k) \leftarrow Q(1:m, 1:k - 1)^t A(1:m, k)$
            \State $z \leftarrow A(1:m, k) - Q(1:m, 1:k - 1) R(1:k - 1, k)$
            \State $R(k, k) \leftarrow \| z \|_2$
            \State $Q(1:m, k) \leftarrow z / R(k, k)$
        \EndFor
    \end{algorithmic}
    \end{solution}

    \question Considere a matriz $A: m \times n$. Sabendo que ao final do processo das transforma\c{c}\~{o}es de Householder, podemos escrever a express\~{a}o: $Q_m \ldots Q_2 Q_1 A = R$.
    \begin{parts}
        \part Em cada etapa $k$: qual o vetor que define a transforma\c{c}\~{a}o $Q_k$? Qual a express\~{a}o dessa matriz? Esta matriz \'{e} armazenada explicitamente? Como s\~{a}o armazenadas as informa\c{c}\~{o}es da matriz $Q_k$ que permite realizar posteriormente um produto da forma $Q_k v$, $v \in \mathbb{R}^m$?
        \begin{solution}
            Na etapa $k$ o vetor que define a transforma\c{c}\~{a}o $Q_k$ \'{e} $A(k:m, k)$ tal que
            \begin{align*}
                Q_k &= \begin{bmatrix}
                    1 & 0 & \ldots & 0 & 0\\
                    0 & 1 & \ldots & 0 & 0 \\
                    \vdots & \vdots & \ddots & \vdots & \vdots \\
                    0 & 0 & \ldots & 1 & 0 \\
                    0 & 0 & \ldots & 0 & I - 2 \left( u u^t \right) / \left( u^t u \right)
                \end{bmatrix},
            \end{align*}
            onde $u = A(k:m, k) \pm \| A(k:m,k) \| e_1$.

            A matriz $Q_k$ n\~{a}o \'{e} armazenada explicitamente. Apenas deve-se armazenar o vetor $u$.
        \end{solution}

        \part Como obter o produto $Q_k v$, $v \in \mathbb{R}^m$ sem calcular explicitamente a matriz $Q$?
        \begin{solution}
            O produto pode ser divido em duas partes da seguinte forma:
            \begin{align*}
                (Q_k v)(1:k - 1) &= v(1:k - 1), \\
                (Q_k v)(k:m) &= (I - 2 (u u^t) / (u^t u)) v(k:m) \\
                &= v(k:m) - (2 u^t v(k:m) / (u^t u)) u.
            \end{align*}
        \end{solution}

        \part Qual a rela\c{c}\~{a}o entre as colunas da matriz $A$ e as colunas da matriz $R$? Justifique.
        \begin{solution}
            As colunas da matriz $R$ correspondem as colunas da matriz $A$ ap\'{o}s o processo de ortogonaliza\c{c}\~{a}o.
        \end{solution}
    \end{parts}

    \question Supor que em cada etapa $k$ a informa\c{c}\~{a}o da matriz $Q_k$ armazenada seja apenas o vetor $z$ que define a transforma\c{c}\~{a}o, no caso, formado pelos elementos $z = \left( a_{k, k}, a_{k + 1, k}, \ldots, a_{m, k} \right)^t$ e a norma-2 deste vetor. Quando for necess\'{a}rio aplicar a transforma\c{c}\~{a}o $Q_k$ sobre um vetor $y$, ser\'{a} preciso realizar: o produto interno $\beta = w^t y$, onde $w = z - v$ e $v = \left( \mp \| z \|_2, 0, \ldots, 0 \right)^t$ e o produto interno $\alpha = w^t w$ e finalmente uma opera\c{c}\~{a}o do tipo: $\hat{y} = y - \gamma w$, onde $\gamma = \left( 2 \beta \right) / \alpha$. Lembrando que apenas o vetor $z$ e sua norma-2 est\~{a}o armazenados, desenvolva f\'{o}rmulas simples para obter estes produtos internos e o vetor $\hat{y}$ somente em fun\c{c}\~{a}o das componentes de $z$ e de sua norma-2.
    \begin{solution}
        Como $z = \left( a_{k, k}, a_{k + 1, k}, \ldots, a_{m, k} \right)^t$ e $v = \left( \mp \| z \|_2, 0, \ldots, 0 \right)^t$ temos que $w = \left( a_{k, k} \pm \| z \|_2, a_{k + 1, k} \ldots, a_{m, k}  \right)^t$. Logo,
        \begin{align*}
            \hat{y} &= y - \gamma w \\
            &= y - \frac{2 \beta}{\alpha} w \\
            &= y - 2 \frac{w^t y}{w^t w} w.
        \end{align*}

        Em rela\c{c}\~{a}o ao produto interno $w^t w$ temos que
        \begin{align*}
            w^t w &= \left( a_{k, k} \pm \| z \|_2 \right)^2 + a_{k + 1, k}^2 + \ldots + a_{m, k}^2 \\
            &= a_{k, k}^2 \pm 2 a_{k, k} \| z \|_2 + \| z \|_2^2 + a_{k + 1, k}^2 + \ldots + a_{m, k}^2 \\
            &= 2 \| z \|_2^2 + 2 a_{k, k} \| z \|_2.
        \end{align*}
    \end{solution}

    \question Considere a matriz $Q = I - 2 u u^t$, $\| u \|_2 = 1$, um refletor de Householder. Calcule os autovalores e o determinante de $Q$.
    \begin{solution}
       Primeiro vamos calcular os autovalores.

       Suponhamos que $u$ \'{e} autovetor de $Q$, ent\~{a}o
       \begin{align*}
           Q u &= \left( I - 2 u u^t \right) u \\
           &= u - 2 u u^t u \\
           &= u - 2 u = -u && \| u \|_2 = 1.
       \end{align*}
       Logo, verificamos que $u$ \'{e} autovetor de $Q$ associado ao autovalor $-1$.

       Suponhamos que um vetor $v$ ortogonal a $u$, i.e., $v^t u = u^t v = 1$, \'{e} autovetor de $Q$, ent\~{a}o
       \begin{align*}
           Q v &= \left( I - 2 u u^t \right) v \\
           &= v - 2 u u^t v \\
           &= v && u^t v = 0.
       \end{align*}
       Logo, verificamos que $v$ \'{e} autovetor de $Q$ associado ao autovalor $1$. Como existem $n - 1$ vetores ortogonais ao vetor $u$ temos que existem $n - 1$ autovalores iguais a $1$.

       Agora vamos calcular o determinante de $Q$. Como o determinante \'{e} o produto dos autovalores da matriz $Q$ temos que $\text{det}(Q) = -1$.
    \end{solution}

    \question Considere o sistema linear $A x = b$,
    \begin{align*}
        A &= \begin{bmatrix}
            1 & 1 \\
            0 & 3 \\
            0 & 4
        \end{bmatrix} \text{ e } b = \begin{bmatrix}
            2 \\
            1 \\
            3
        \end{bmatrix}.
    \end{align*}
    \begin{parts}
        \part Usando transforma\c{c}\~{o}es de Householder obtenha a proje\c{c}\~{a}o ortogonal de $b$ em $\text{Im}(A)$.
        \begin{solution}
            A proje\c{c}\~{a}o ortognal de $b$ em $\text{Im}(A)$ corresponde a $A x$, onde $x$ \'{e} a solu\c{c}\~{a}o do problema de Quadrados M\'{i}nimos $A x = b$. Vamos resolver o problema de Quadrados M\'{i}nimos utilizando transforma\c{c}\~{o}es de Householder.
            \begin{align*}
                Q A &= \begin{bmatrix}
                    1 & 0 & 0 \\
                    0 & -0.60 & -0.80 \\
                    0 & -0.80 & 0.60
                \end{bmatrix} \begin{bmatrix}
                    1 & 1 \\
                    0 & 3 \\
                    0 & 4
                \end{bmatrix} = \begin{bmatrix}
                    1 & 1 \\
                    0 & -5 \\
                    0 & 0
                \end{bmatrix}, \\
                Q b &= \begin{bmatrix}
                    1 & 0 & 0 \\
                    0 & -0.60 & -0.80 \\
                    0 & -0.80 & 0.60
                \end{bmatrix} \begin{bmatrix}
                    2 \\
                    1 \\
                    3
                \end{bmatrix} = \begin{bmatrix}
                    2 \\
                    -3 \\
                    1
                \end{bmatrix}.
            \end{align*}
            Resolvendo o sistema linear dado por
            \begin{align*}
                \begin{bmatrix}
                    1 & 1 \\
                    0 & -5
                \end{bmatrix} x &= \begin{bmatrix}
                    2 \\
                    -3
                \end{bmatrix}
            \end{align*}
            temos como solu\c{c}\~{a}o $x_1 = 7 / 5$ e $x_2 = 3 / 5$. Logo, a proje\c{c}\~{a}o ortogonal de $b \in \text{Im}(A)$ corresponde ao vetor $(2, 9 / 5, 12 / 5)^t$.
        \end{solution}

        \part Obtenha a proje\c{c}\~{a}o ortogonal de $b$ em $\text{N}(A^t)$.
        \begin{solution}
            Temos que $\text{N}(A^t)$ correspondem aos vetores $x$ tal que
            \begin{align*}
                \begin{bmatrix}
                    1 & 0 & 0 \\
                    1 & 3 & 4
                \end{bmatrix} x &= \begin{bmatrix}
                    0 \\
                    0
                \end{bmatrix}.
            \end{align*}
            Verifica-se que os vetores $x$ que satisfazem o sistema acima s\~{a}o da forma $\alpha (0, -4, 3)^t$. Logo, podemos utilizar o vetor $u / \| u \|$, $u = (0, -4, 3)^t$, para construir o projetor ortogonal $u u^t / (u^t u)$ e calcular a proje\c{c}\~{a}o de $b$ em $u$. Portanto,
            \begin{align*}
                (u u^t / (u^t u)) ) b &= ((u^t b) / (u^t u)) u =  (5 / 25) u = u / 5.
            \end{align*}
        \end{solution}
    \end{parts}

    \question A fatora\c{c}\~{a}o QR de uma matriz $A :  m \times n$ pode ser obtida atrav\'{e}s das Rota\c{c}\~{o}es de Givens. Pesquisar este assunto no livro do Watkins\nocite{Watkins:2004:fundamentals} e/ou Golub\nocite{Golub:1996:matrix}: matriz de rota\c{c}\~{a}o, caso $2 \times 2$ e como aplicar as rota\c{c}\~{o}es sobre uma matriz geral $A : m \times n$. Detalhar como s\~{a}o as matrizes $Q_k$ de cada etapa da transforma\c{c}\~{a}o. Como aplicar as transforma\c{c}\~{o}es para resolver um sistema linear $n \times n$ ou um problema de Quadrados M\'{i}nimos.
    \begin{solution}
        Retirado do Golub, p\'{a}gina 208.

        Uma matriz ortogonal $Q : 2 \times 2$ \'{e} uma rota\c{c}\~{a}o se tem a forma
        \begin{align*}
            Q = \begin{bmatrix}
                \cos \theta & \sin \theta \\
                -\sin \theta & \cos \theta
            \end{bmatrix}.
        \end{align*}
        Se $y = Q^t x$ ent\~{a}o $y$ \'{e} obtido por uma rota\c{c}\~{a}o de $x$, no sentido hor\'{a}rio, em um \^{a}gulo $\theta$.

        Retirado do Golub, p\'{a}gina 215.

        Uma matriz ortogonal $Q : n \times n$ \'{e} uma rota\c{c}\~{a}o se tem a forma
        \begin{align*}
            Q = \begin{bmatrix}
                1      & \ldots & 0      & \ldots & 0       & \ldots & 0       \\
                \vdots & \ddots & \vdots & \ddots & \vdots  & \ldots & \vdots  \\
                0      & \ldots & c      & \ldots &  s      & \ldots & 0       \\
                \vdots & \ddots & \vdots & \ddots & \vdots  & \ldots & \vdots  \\
                0      & \ldots & -s     & \ldots &  c      & \ldots & 0       \\
                \vdots & \ddots & \vdots & \ddots & \vdots  & \ldots & \vdots  \\
                0      & \ldots & 0      & \ldots & 0       & \ldots & 1 
            \end{bmatrix},
        \end{align*}
        onde $c = \cos \theta$ e $s = \sin \theta$ para algum $\theta$.

        Se $x \in \mathbb{R}^n$ e $y = Q^t x$, ent\~{a}o
        \begin{align*}
            y_j = \begin{cases}
                c x_i - s x_k, & j = i, \\
                s x_i + c x_k, & j = k, \\
                x_j, & j \neq i, k.
            \end{cases}
        \end{align*}
        Ent\~{a}o podemos for\c{c}ar $y_k$ ser zero ao fazermos
        \begin{align*}
            c &= \frac{x_i}{\sqrt{x_i^2 + x_k^2}} \text{ e } s = \frac{-x_k}{\sqrt{x_i^2 + x_k^2}}.
        \end{align*}
    \end{solution}

    \question Resolva o sistema linear $A x = b$,
    \begin{align*}
        A = \begin{bmatrix}
            1 & 1 & 2 \\
            0 & 0 & 1 \\
            1 & 0 & 0
        \end{bmatrix} \text{ e } b = \begin{bmatrix}
            2 \\
            1 \\
            1
        \end{bmatrix},
    \end{align*}
    atrav\'{e}s da fatora\c{c}\~{a}o QR, usando rota\c{c}\~{o}es de Givens.
    \begin{solution}
        Para resolver o sistema linear por rota\c{c}\~{o}es de Givens temos a seguinte sequencia de rota\c{c}\~{o}es:
        \begin{enumerate}
            \item zerar o elemento $a_{31}$:

                Precisamos que
                \begin{align*}
                    c &= a_{11} / \left( a_{11}^2 + a_{31}^2 \right)^{1 / 2} \\
                    &= 1 / \left( 2 \right)^{1 / 2}, \\
                    s &= -a_{31} / \left( a_{11}^2 + a_{31}^2 \right)^{1 / 2} \\
                    &= -1 / \left( 2 \right)^{1 / 2}.
                \end{align*}
                Ent\~{a}o
                \begin{align*}
                    Q &= \begin{bmatrix}
                        0.71 & 0.00 & -0.71 \\
                        0.00 & 1.00 & 0.00 \\
                        0.71 & 0.00 & 0.71
                    \end{bmatrix}, \\
                    Q^t A &= \begin{bmatrix}
                        1.41 & 0.71 & 1.41 \\
                        0.00 & 0.00 & 1.00 \\
                        0.00 & -0.71 & -1.41
                    \end{bmatrix}.
                \end{align*}

            \item permutar as linhas $2$ e $3$:

                Ent\~{a}o
                \begin{align*}
                    P &= \begin{bmatrix}
                        1 & 0 & 0 \\
                        0 & 0 & 1 \\
                        0 & 1 & 0
                    \end{bmatrix}, \\
                    P \left( Q^t A \right) &= \begin{bmatrix}
                        1.41 & 0.71 & 1.41 \\
                        0.00 & -0.71 & -1.41 \\
                        0.00 & 0.00 & 1.00
                    \end{bmatrix}.
                \end{align*}
        \end{enumerate}

        Agora efetuando a mesma rota\c{c}\~{a}o e permuta\c{c}\~{a}o no vetor $b$, temos
        \begin{align*}
            P \left( Q^t b \right) &= \begin{bmatrix}
                1 & 0 & 0 \\
                0 & 0 & 1 \\
                0 & 1 & 0
            \end{bmatrix} \begin{bmatrix}
                2.12 \\
                1.00 \\
                -0.71
            \end{bmatrix} \\
            &= \begin{bmatrix}
                2.12 \\
                -0.71 \\
                1.00
            \end{bmatrix}.
        \end{align*}

        Por fim, temos que resolver o sistema linear
        \begin{align*}
            \begin{bmatrix}
                1.41 & 0.71 & 1.41 \\
                0.00 & -0.71 & -1.41 \\
                0.00 & 0.00 & 1.00
            \end{bmatrix} x &= \begin{bmatrix}
                2.12 \\
                -0.71 \\
                1.00
            \end{bmatrix}, \\
            x &= \begin{bmatrix}
                1 \\
                -1 \\
                1
            \end{bmatrix}.
        \end{align*}
    \end{solution}

    \question Obtenha a solu\c{c}\~{a}o de Quadrados M\'{i}nimos para o sistema linear $A x = b$,
    \begin{align*}
        A = \begin{bmatrix}
            1 & 5 & 5 \\
            2 & 3 & 7 \\
            0 & 1 & 6 \\
            0 & 7 & 9
        \end{bmatrix} \text{ e } b = \begin{bmatrix}
            1 \\
            -1 \\
            2 \\
            -2
        \end{bmatrix},
    \end{align*}
    usando o processo de fatora\c{c}\~{a}o ortogonal obtida atrav\'{e}s de rota\c{c}\~{o}es de Givens.
    \begin{solution}
        Para resolver o problema de Quadrados M\'{i}nimos por rota\c{c}\~{o}es de Givens temos a seguinte sequencia de rota\c{c}\~{o}es:
        \begin{enumerate}
            \item zerar o elemento $a_{21}$:

                Precisamos que
                \begin{align*}
                    c &= a_{11} / \left( a_{11}^2 + a_{21}^2 \right)^{1 / 2} \\
                    &= 1 / \left( 5 \right)^{1 / 2}, \\
                    s &= - a_{21} / \left( a_{11}^2 + a_{21}^2 \right)^{1 / 2} \\
                    &= -2 / \left( 5 \right)^{1 / 2}.
                \end{align*}
                Ent\~{a}o
                \begin{align*}
                    Q_1 &= \begin{bmatrix}
                        0.45 & -0.89 & 0.00 & 0.00 \\
                        0.89 & 0.45 & 0.00 & 0.00 \\
                        0.00 & 0.00 & 1.00 & 0.00 \\
                        0.00 & 0.00 & 0.00 & 1.00
                    \end{bmatrix}, \\
                    Q_1^t A &= \begin{bmatrix}
                        2.24 & 4.92 & 8.50 \\
                        0.00 & -3.13 & -1.34 \\
                        0.00 & 1.00 & 6.00 \\
                        0.00 & 7.00 & 9.00
                    \end{bmatrix}, \\
                    Q_1^t b &= \begin{bmatrix}
                        -0.44 \\
                        -1.34 \\
                        2.00 \\
                        -2.00 \\
                    \end{bmatrix}.
                \end{align*}

            \item zerar o elemento $a_{32}$:

                Precisamos que
                \begin{align*}
                    c &= a_{22} / \left( a_{22}^2 + a_{32}^2 \right)^{1 / 2} \\
                    &= -3.13 / \left( 3.13^2 + 1^2 \right)^{1 / 2}, \\
                    s &= - a_{32} / \left( a_{22}^2 + a_{32}^2 \right)^{1 / 2} \\
                    &= -1 / \left( 3.13^2 + 1^2 \right)^{1 / 2}.
                \end{align*}
                Ent\~{a}o
                \begin{align*}
                    Q_2 &= \begin{bmatrix}
                        1.00 & 0.00 & 0.00 & 0.00 \\
                        0.00 & -0.95 & -0.30 & 0.00 \\
                        0.00 & 0.30 & -0.95 & 0.00 \\
                        0.00 & 0.00 & 0.00 & 1.00
                    \end{bmatrix}, \\
                    Q_2^t Q_1^t A &= \begin{bmatrix}
                        2.24 & 4.92 & 8.50 \\
                        0.00 & 3.29 & 3.10 \\
                        0.00 & 0.00 & -5.31 \\
                        0.00 & 7.00 & 9.00
                    \end{bmatrix}, \\
                    Q_2^t Q_1^t b &= \begin{bmatrix}
                        -0.44 \\
                        1.87 \\
                        -1.50 \\
                        -2.00
                    \end{bmatrix}.
                \end{align*}

            \item zerar o elemento $a_{42}$:

                Precisamos que
                \begin{align*}
                    c &= a_{22} / \left( a_{22}^2 + a_{42}^2 \right)^{1 / 2} \\
                    &= 3.29 / \left( 3.29^2 + 7^2 \right)^{1 / 2}, \\
                    s &= - a_{42} / \left( a_{22}^2 + a_{42}^2 \right)^{1 / 2} \\
                    &= -7 / \left( 3.29^2 + 7^2 \right)^{1 / 2}.
                \end{align*}
                Ent\~{a}o
                \begin{align*}
                    Q_3 &= \begin{bmatrix}
                        1.00 & 0.00 & 0.00 & 0.00 \\
                        0.00 & 0.43 & 0.00 & -0.91 \\
                        0.00 & 0.00 & 1.00 & 0.00 \\
                        0.00 & 0.91 & 0.00 & 0.43
                    \end{bmatrix}, \\
                    Q_3^t Q_2^t Q_1^t A &= \begin{bmatrix}
                        2.24 & 4.92 & 8.50 \\
                        0.00 & 7.78 & 9.52 \\
                        0.00 & 0.00 & -5.31 \\
                        0.00 & 0.00 & 1.05
                    \end{bmatrix}, \\
                    Q_3^t Q_2^t Q_1^t b &= \begin{bmatrix}
                        -0.44 \\
                        -1.01 \\
                        -1.50 \\
                        -2.56
                    \end{bmatrix}.
                \end{align*}

            \item zerar o elemento $a_{43}$;

                Precisamos que
                \begin{align*}
                    c &= a_{33} / \left( a_{33}^2 + a_{43}^2 \right)^{1 / 2} \\
                    &= -5.31 / \left( -5.31^2 + 1.05^2 \right)^{1 / 2}, \\
                    s &= - a_{43} / \left( a_{33}^2 + a_{43}^2 \right)^{1 / 2} \\
                    &= -1.05 / \left( -5.31^2 + 1.05^2 \right)^{1 / 2}.
                \end{align*}
                Ent\~{a}o
                \begin{align*}
                    Q_4 &= \begin{bmatrix}
                        1.00 & 0.00 & 0.00 & 0.00 \\
                        0.00 & 1.00 & 0.00 & 0.00 \\
                        0.00 & 0.00 & -0.98 & -0.19 \\
                        0.00 & 0.00 & 0.19 & -0.98
                    \end{bmatrix}, \\
                    Q_4^t Q_3^t Q_2^t Q_1^t A &= \begin{bmatrix}
                        2.24 & 4.92 & 8.50 \\
                        0.00 & 7.78 & 9.52 \\
                        0.00 & 0.00 & 5.40 \\
                        0.00 & 0.00 & 0.00
                    \end{bmatrix}, \\
                    Q_4^t Q_3^t Q_2^t Q_1^t b &= \begin{bmatrix}
                        -0.44 \\
                        -1.01 \\
                        0.98 \\
                        2.80
                    \end{bmatrix}.
                \end{align*}
        \end{enumerate}
        Agora resolvemos o sistema linear representado por
        \begin{align*}
            \begin{bmatrix}
                2.24 & 4.92 & 8.50 \\
                0.00 & 7.78 & 9.52 \\
                0.00 & 0.00 & 5.40
            \end{bmatrix} \begin{bmatrix}
                x_1 \\
                x_2 \\
                x_3
            \end{bmatrix} &= \begin{bmatrix}
                -0.44 \\
                -1.01 \\
                0.98
            \end{bmatrix},
        \end{align*}
        cuja solu\c{c}\~{a}o \'{e} $x_1 = -0.11$, $x_2 = -0.35$ e $x_3 = 0.18$.
    \end{solution}

    \question
    \begin{parts}
        \part Seja a matriz $J$ uma matriz dada por
        \begin{align*}
            J &= \begin{bmatrix}
                \cos(\theta) & \sin(\theta) \\
                -\sin(\theta) & \cos(\theta)
            \end{bmatrix}.
        \end{align*}
        Mostre que $J$ \'{e} ortogonal. D\^{e} uma interpreta\c{c}\~{a}o geom\'{e}trica para a a\c{c}\~{a}o de $J$ sobre vetores de $\mathbb{R}^2$.
        \begin{solution}
            Se $J$ \'{e} ortogonal ent\~{a}o $J^t J = J J^t = I$. Ent\~{a}o
            \begin{align*}
                J J^t &= \begin{bmatrix}
                    \cos(\theta) & \sin(\theta) \\
                    -\sin(\theta) & \cos(\theta)
                \end{bmatrix} \begin{bmatrix}
                    \cos(\theta) & -\sin(\theta) \\
                    \sin(\theta) & \cos(\theta)
                \end{bmatrix} \\
                &= \begin{bmatrix}
                    \cos^2(\theta) + \sin^2(\theta) & -\cos(\theta)\sin(\theta) + \sin(\theta)\cos(\theta) \\
                    -\sin(\theta)\cos(\theta) + \cos(\theta)\sin(\theta) & \sin^2(\theta) + \cos^2(\theta)
                \end{bmatrix} \\
                &= \begin{bmatrix}
                    1 & 0 \\
                    0 & 1
                \end{bmatrix} = I.
            \end{align*}

            Para o $\mathbb{R}^2$ temos
            \begin{center}
                \begin{tikzpicture}[scale=2]
                    \draw[->] (-1.2, 0) -- (1.2, 0) node[below right] {$x_1$};
                    \draw[->] (0, -1.2) -- (0, 1.2) node[above right] {$x_2$};
                    \draw[->] (0, 0) -- (1, 0) node[above] {$e_1$};

                    \draw[->, blue] (0, 0) -- (0.86, -0.5) node[right] {$J(\pi / 6) e_1$};
                    \draw[blue] (0.2, 0) arc (0:-30:0.2) node[right] {$\theta = 30^o$};

                    \draw[->, red] (0, 0) -- (0.5, -0.86) node[below right] {$J(\pi / 6) e_1$};
                    \draw[red] (0.6, 0) arc (0:-60:0.6) node[below right] {$\theta = 60^o$};
                \end{tikzpicture}
            \end{center}
        \end{solution}

        \part Idem para a matriz
        \begin{align*}
            F &= \begin{bmatrix}
                -\cos(\theta) & \sin(\theta) \\
                \sin(\theta) & \cos(\theta)
            \end{bmatrix}.
        \end{align*}
        \begin{solution}
            Se $J$ \'{e} ortogonal ent\~{a}o $J^t J = J J^t = I$. Ent\~{a}o
            \begin{align*}
                J J^t &= \begin{bmatrix}
                    -\cos(\theta) & \sin(\theta) \\
                    \sin(\theta) & \cos(\theta)
                \end{bmatrix} \begin{bmatrix}
                    -\cos(\theta) & \sin(\theta) \\
                    \sin(\theta) & \cos(\theta)
                \end{bmatrix} \\
                &= \begin{bmatrix}
                    \cos^2(\theta) + \sin^2(\theta) & -\cos(\theta)\sin(\theta) + \sin(\theta)\cos(\theta) \\
                    -\sin(\theta)\cos(\theta) + \cos(\theta)\sin(\theta) & \sin^2(\theta) + \cos^2(\theta)
                \end{bmatrix} \\
                &= \begin{bmatrix}
                    1 & 0 \\
                    0 & 1
                \end{bmatrix} = I.
            \end{align*}

            Para o $\mathbb{R}^2$ temos
            \begin{center}
                \begin{tikzpicture}[scale=2]
                    \draw[->] (-1.2, 0) -- (1.2, 0) node[below right] {$x_1$};
                    \draw[->] (0, -1.2) -- (0, 1.2) node[above right] {$x_2$};
                    \draw[->] (0, 0) -- (1, 0) node[above] {$e_1$};

                    \draw[->, blue] (0, 0) -- (-0.86, 0.5) node[above left] {$J(\pi / 6) e_1$};
                    \draw[blue] (-0.2, 0) arc (0:-30:-0.2) node[below left] {$\theta = 30^o$};

                    \draw[->, red] (0, 0) -- (-0.5, 0.86) node[above left] {$J(\pi / 6) e_1$};
                    \draw[red] (-0.4, 0) arc (0:-60:-0.4) node[left] {$\theta = 60^o$};
                \end{tikzpicture}
            \end{center}
        \end{solution}

        \part Idem para a matriz
        \begin{align*}
            G &= \begin{bmatrix}
                \cos(\theta) & -\sin(\theta) \\
                \sin(\theta) & \cos(\theta)
            \end{bmatrix}.
        \end{align*}
        \begin{solution}
            Se $J$ \'{e} ortogonal ent\~{a}o $J^t J = J J^t = I$. Ent\~{a}o
            \begin{align*}
                J J^t &= \begin{bmatrix}
                    -\cos(\theta) & \sin(\theta) \\
                    \sin(\theta) & \cos(\theta)
                \end{bmatrix} \begin{bmatrix}
                    -\cos(\theta) & \sin(\theta) \\
                    \sin(\theta) & \cos(\theta)
                \end{bmatrix} \\
                &= \begin{bmatrix}
                    \cos^2(\theta) + \sin^2(\theta) & -\cos(\theta)\sin(\theta) + \sin(\theta)\cos(\theta) \\
                    -\sin(\theta)\cos(\theta) + \cos(\theta)\sin(\theta) & \sin^2(\theta) + \cos^2(\theta)
                \end{bmatrix} \\
                &= \begin{bmatrix}
                    1 & 0 \\
                    0 & 1
                \end{bmatrix} = I.
            \end{align*}

            Para o $\mathbb{R}^2$ temos
            \begin{center}
                \begin{tikzpicture}[scale=2]
                    \draw[->] (-1.2, 0) -- (1.2, 0) node[below right] {$x_1$};
                    \draw[->] (0, -1.2) -- (0, 1.2) node[above right] {$x_2$};
                    \draw[->] (0, 0) -- (1, 0) node[above] {$e_1$};

                    \draw[->, blue] (0, 0) -- (0.86, 0.5) node[above right] {$J(\pi / 6) e_1$};
                    \draw[blue] (0.2, 0) arc (0:30:0.2) node[right] {$\theta = 30^o$};

                    \draw[->, red] (0, 0) -- (0.5, 0.86) node[above right] {$J(\pi / 6) e_1$};
                    \draw[red] (0.6, 0) arc (0:60:0.6) node[right] {$\theta = 60^o$};
                \end{tikzpicture}
            \end{center}
        \end{solution}
    \end{parts}

    \question Considere $A : n \times n$, e $\text{posto}(A) = n$. Demonstre que o fator $R$ da fatora\c{c}\~{a}o QR de $A$ ($Q$ ortogonal e $R$ triangular superior com elementos positivos na diagonal) \'{e} o fator de Cholesky da matriz $A^t A$.
    \begin{solution}
        Seja $A = Q R$, ent\~{a}o
        \begin{align*}
            A^t A &= \left( R^t Q^t \right) \left( Q R \right) \\
            &= R^t R && \text{pois $Q$ \'{e} ortogonal}.
        \end{align*}
        Logo, a matriz $R$ da fatora\c{c}\~{a}o QR corresponde ao fator de Cholesky da matriz $A^t A$.
    \end{solution}

    \question Seja $A$ uma matriz com a propriedade que as colunas \'{i}mpares s\~{a}o ortogonais \`{a}s colunas pares. Na fatora\c{c}\~{a}o ortogonal reduzida de $A$, $A = \hat{Q} \hat{R}$, qual a estrutura especial da matriz $\hat{R}$? Justifique.
    \begin{solution}
        Seja
        \begin{align*}
            A &= \begin{bmatrix}
                A_{1 \mdot} & A_{2 \mdot} & A_{3 \mdot} & A_{4 \mdot} & \ldots & A_{m \mdot}
            \end{bmatrix}
        \end{align*}
        tal que $A_{i \mdot}^t A_{j \mdot} = 0$ para todo $i = 2k$ e $j = 2k + 1$, $k \in \mathbb{N}$.

        Ao aplicarmos o processo de ortogonaliza\c{c}\~{a}o de Gram-Schmidt na matriz $A$ obtemos uma matriz ortogonal $Q$ dada por
        \begin{align*}
            V_{1 \mdot} &\leftarrow A_{1 \mdot}, \\
            Q_{1 \mdot} &\leftarrow V_{1 \mdot} / \| V_{1 \mdot} \|, \\
            V_{2 \mdot} &\leftarrow A_{2 \mdot} - \left( Q_{1 \mdot}^t A_{2 \mdot} \right) Q_{1 \mdot} = A_{2 \mdot}, \\
            Q_{2 \mdot} &\leftarrow V_{2 \mdot} / \| V_{2 \mdot}, \\
            V_{3 \mdot} &\leftarrow A_{3 \mdot} - \left( Q_{1 \mdot}^t A_{3 \mdot} \right) Q_{1 \mdot} - \left( Q_{2 \mdot}^t A_{3 \mdot} \right) Q_{2 \mdot} \\
            &\quad = A_{3 \mdot} - \left( Q_{1 \mdot}^t A_{3} \mdot \right) Q_{1 \mdot}, \\
            Q_{3 \mdot} &\leftarrow V_{3 \mdot} / \| V_{3 \mdot} \|, \\
            V_{4 \mdot} &\leftarrow A_{4 \mdot} - \left( Q_{1 \mdot}^t A_{4 \mdot} \right) Q_{1 \mdot} - \left( Q_{2 \mdot}^t A_{4 \mdot} \right) - \left( Q_{3 \mdot}^t A_{4 \mdot} \right) Q_{3 \mdot} \\
            &\quad = A_{4 \mdot} - \left( V_{2 \mdot}^t A_{4 \mdot} \right) V_{2 \mdot}, \\
            Q_{4 \mdot} &\leftarrow Q_{4 \mdot} / \| Q_{4 \mdot} \|, \\
            \vdots \\
            V_{i \mdot} &\leftarrow A_{i \mdot} - \sum_{k = 1}^{i / 2} \left( Q_{2k \mdot}^t A_{i \mdot} \right) Q_{2k \mdot}, \\
            Q_{i \mdot} &\leftarrow V_{i \mdot} / \| V_{i \mdot} \|, \\
            V_{j \mdot} &\leftarrow A_{j \mdot} - \sum_{k = 1}^{(j - 1) / 2} \left( Q_{2k + 1 \mdot}^t A_{j \mdot} \right) Q_{2k + 1 \mdot}, \\
            Q_{j \mdot} &\leftarrow V_{j \mdot} / \| V_{j \mdot} \|
        \end{align*}
        e uma matriz $R$ dada por
        \begin{align*}
            R &= \begin{bmatrix}
                \| V_{1 \mdot} \| & Q_{1 \mdot}^t A_{2 \mdot} & Q_{1 \mdot}^t A_{3 \mdot} & Q_{1 \mdot}^t A_{4 \mdot} & \ldots & Q_{1 \mdot}^t A_{i \mdot} & Q_{1 \mdot}^t A_{j \mdot} & \ldots \\  
                0 & \| V_{2 \mdot} \| & Q_{2 \mdot}^t A_{3 \mdot} & Q_{2 \mdot}^t A_{4 \mdot} & \ldots & Q_{2 \mdot}^t A_{i \mdot} & Q_{2 \mdot}^t A_{j \mdot} & \ldots \\ 
                0 &  0 & \| V_{3 \mdot} \| & Q_{3 \mdot}^t A_{4 \mdot} & \ldots & Q_{3 \mdot}^t A_{i \mdot} & Q_{3 \mdot}^t A_{j \mdot} & \ldots \\ 
                0 &  0 & 0 & \| V_{4 \mdot} \| &  \ldots & Q_{4 \mdot} A_{i \mdot} & Q_{4 \mdot} A_{j \mdot} & \ldots \\ 
                \vdots & \vdots & \vdots & \vdots & \ddots & \vdots & \vdots & \ddots \\
                0 &  0 & 0 & 0 & \ldots \| V_{i \mdot} \| & Q_{i \mdot} A_{j \mdot} & \ldots \\
                0 &  0 & 0 & 0 & \ldots  0 & \| V_{j \mdot} \|  & \ldots \\
                \vdots & \vdots & \vdots & \vdots & \ddots & \vdots & \vdots & \ddots 
            \end{bmatrix} \\
            &= \begin{bmatrix}
                \| V_{1 \mdot} \| & 0 & Q_{1 \mdot}^t A_{3 \mdot} & 0 & \ldots & 0 & Q_{1 \mdot}^t A_{j \mdot} & \ldots \\  
                0 & \| V_{2 \mdot} \| & 0 & Q_{2 \mdot}^t A_{4 \mdot} & \ldots & Q_{2 \mdot}^t A_{i \mdot} & 0 & \ldots \\ 
                0 &  0 & \| V_{3 \mdot} \| & 0 & \ldots & 0 & Q_{3 \mdot}^t A_{j \mdot} & \ldots \\ 
                0 &  0 & 0 & \| V_{4 \mdot} \| &  \ldots & Q_{4 \mdot} A_{i \mdot} & 0 & \ldots \\ 
                \vdots & \vdots & \vdots & \vdots & \ddots & \vdots & \vdots & \ddots \\
                0 &  0 & 0 & 0 & \ldots \| V_{i \mdot} \| & 0 & \ldots \\
                0 &  0 & 0 & 0 & \ldots  0 & \| V_{j \mdot} \|  & \ldots \\
                \vdots & \vdots & \vdots & \vdots & \ddots & \vdots & \vdots & \ddots 
            \end{bmatrix}.
        \end{align*}
    \end{solution}
\end{questions}
\bibliographystyle{plain}
\bibliography{bibliography}
\end{document}
