% Filename: lista13.tex
% 
% This code is part of 'Solutions for MT402, Matrizes'
% 
% Description: This file corresponds to the solutions of homework sheet 13.
% 
% Created: 16.06.12 08:15:39 AM
% Last Change: 29.06.12 05:39:45 PM
% 
% Authors:
% - Raniere Silva (2012): initial version
%
% Copyright (c) 2012 Raniere Silva <r.gaia.cs@gmail.com>
% 
% This work is licensed under the Creative Commons Attribution-ShareAlike 3.0 Unported License. To view a copy of this license, visit http://creativecommons.org/licenses/by-sa/3.0/ or send a letter to Creative Commons, 444 Castro Street, Suite 900, Mountain View, California, 94041, USA.
%
% This work is distributed in the hope that it will be useful, but WITHOUT ANY WARRANTY; without even the implied warranty of MERCHANTABILITY or FITNESS FOR A PARTICULAR PURPOSE.
%
\documentclass[a4paper,12pt, leqno, answers]{exam}
% Customiza\c{c}\~{a}o da classe exam
\newcommand{\mycheader}{Lista 13 - SVD}
\header{MT402}{\mycheader}{\thepage/\numpages}
\headrule
\footer{Dispon\'{i}vel em \\\url{https://github.com/r-gaia-cs/solucoes_lista_matrizes}
}{}{Reportar erros para \\% Filename: maintainer.tex
% 
% This code is part of 'Solutions for MT402, Matrizes'
% 
% Description: This file keeps the email of the mainteiner.
% 
% Created: 08.05.12 09:46:55 PM
% Last Change: 04.06.12 10:42:01 PM
% 
% Authors:
% - Raniere Silva (2012): initial version
% 
% Copyright (c) 2012 Raniere Silva <r.gaia.cs@gmail.com>
% 
% This work is licensed under the Creative Commons Attribution-ShareAlike 3.0 Unported License. To view a copy of this license, visit http://creativecommons.org/licenses/by-sa/3.0/ or send a letter to Creative Commons, 444 Castro Street, Suite 900, Mountain View, California, 94041, USA.
%
% This work is distributed in the hope that it will be useful, but WITHOUT ANY WARRANTY; without even the implied warranty of MERCHANTABILITY or FITNESS FOR A PARTICULAR PURPOSE.
%
\href{mailto:r.gaia.cs@gmail.com}{r.gaia.cs@gmail.com}
}
\footrule 
\pagestyle{headandfoot}
\renewcommand{\solutiontitle}{\noindent\textbf{Solu\c{c}\~{a}o:}\enspace}
\SolutionEmphasis{\slshape}
\unframedsolutions
\pointname{}

% Filename: paper_size.tex
% 
% This code is part of 'Solutions for MT402, Matrizes'
% 
% Description: This file corresponds to the paper size output.
% 
% Created: 08.05.12 09:46:55 PM
% Last Change: 04.06.12 10:42:01 PM
% 
% Authors:
% - Raniere Silva (2012): initial version
% 
% Copyright (c) 2012 Raniere Silva <r.gaia.cs@gmail.com>
% 
% This work is licensed under the Creative Commons Attribution-ShareAlike 3.0 Unported License. To view a copy of this license, visit http://creativecommons.org/licenses/by-sa/3.0/ or send a letter to Creative Commons, 444 Castro Street, Suite 900, Mountain View, California, 94041, USA.
%
% This work is distributed in the hope that it will be useful, but WITHOUT ANY WARRANTY; without even the implied warranty of MERCHANTABILITY or FITNESS FOR A PARTICULAR PURPOSE.
%
% Para impress\~{a}o
\usepackage[top=3cm, bottom=3cm, left=2cm, right=2cm]{geometry}

% Para ereaders (Kindle, Nook, Kobo, ...) and tablets (iPad, GalaxyTab, ...)
% \usepackage[papersize={180mm,240mm},margin=2mm]{geometry}
% \sloppy


% Pacotes
\usepackage[utf8]{inputenc}
\usepackage[T1]{fontenc}
\usepackage[brazil]{babel}
\usepackage{amsmath}
\usepackage{amsfonts}
\usepackage{amssymb}
\usepackage{hyperref}
\usepackage{graphicx}

% Customiza\c{c}\~{a}o do pacote amsmath
\allowdisplaybreaks[4]

% Novos ambientes
% \newenvironment{fwsolution}{\begin{EnvFullwidth}\begin{TheSolution}}{\end{TheSolution}\end{EnvFullwidth}}

% Novos comandos
% \newcommand{\mdot}{\text{\LARGE $\boldsymbol{\cdot}$}}
\newcommand{\mdot}{\bullet}
\newtheorem{defi}{Defini\c{c}\~{a}o}
\newtheorem{theorem}{Teorema}

\begin{document}
%cover
\thispagestyle{empty}
\input{cover.tex}
\newpage
\setcounter{page}{1}
\begin{theorem}[Norma matricial induzida\nocite{Meyer:2000:matrix}]
    Uma norma vetorial definida em $\mathbb{C}^p$, $p = m, n$, induz uma norma matricial definida em $\mathbb{C}^{m \times n}$ por
    \begin{align*}
        \| A \| &= \max_{\| x \| = 1} \| A x \|, \forall A \in \mathbb{C}^{m \times n}, x \in \mathbb{C}^{n \times n}.
    \end{align*}
    Para uma norma matricial induzida vale
    \begin{align*}
        \| A x \| &\leq \| A \| \| x \|
    \end{align*}
    e se $A$ for n\~{a}o singular
    \begin{align*}
        \min_{\| x \| = 1} \| A x \| = 1 / \| A^{-1} \|.
    \end{align*}
\end{theorem}
\begin{theorem}[Propriedades da norma-2\nocite{Meyer:2000:matrix}]
    Na norma matricial induzida pela norma-2 vetorial vale que
    \begin{align*}
        \| U^* A V \|_2 = \| A \|_2
    \end{align*}
    quando $U$ e $V$ s\~{a}o matrizes unit\'{a}rias.
\end{theorem}
\begin{defi}[Posto\nocite{Meyer:2000:matrix}]
    O posto de uma matriz $A : m \times n$ \'{e} precisamente a ordem da maior submatriz matriz quadrada n\~{a}o-singular de $A$. Em outras palavras, se $\mathrm{posto}(A) = r$ ent\~{a}o existe pelo menos uma submatriz $r \times r$ n\~{a}o singular em $A$ e n\~{a}o existe nenhuma outra submatriz n\~{a}o-singular de ordem maior que $r$.
\end{defi}
\begin{theorem}[4.2.1 do Watkins\nocite{Watkins:2004:fundamentals}]
    Seja $A \in \mathbb{R}^{n \times n}$ com os seguintes valores singulares: $\sigma_1 \geq \sigma_2 \geq \ldots \geq 0$. Ent\~{a}o 
    \begin{align*}
        \| A \|_A = \max_{x \neq 0} \frac{\| A x \|_2}{\| x \|_2} = \sigma_1.
    \end{align*}
\end{theorem}
\begin{theorem}[Exerc\'{i}cio 4.2.5 do Watkins\nocite{Watkins:2004:fundamentals}]
    Seja $A \in \mathbb{R}^{n \times n}$ com os seguintes valores singulares: $\sigma_1 \geq \sigma_2 \geq \ldots \geq \sigma_n > 0$. Ent\~{a}o 
    \begin{align*}
        \min_{x \neq 0} \frac{\| A x \|_2}{\| x \|_2} = \sigma_n.
    \end{align*}
\end{theorem}
\begin{theorem}[4.5.9 do Meyer\nocite{Meyer:2000:matrix}]
    Se $A$ e $E$ s\~{a}o matrizes $m \times n$ tal que $E$ possue entradas suficientemente pequenas ent\~{a}o
    \begin{align*}
        \mathrm{posto}(A + E) \geq \mathrm{posto}(A).
    \end{align*}
\end{theorem}
\begin{questions}
    \question Considere $A : m \times n$, $\textrm{posto}(A) = r < \min\left\{ m, n \right\}$. Usando a SVD de $A$, mostre que para cada $\epsilon > 0$, existe uma matriz de posto completo $A_\epsilon : m \times n$ tal que $\| A - A_\epsilon \|_2 < \epsilon$.
    \begin{parts}
        \part[Exerc\'{i}cio 4.2.14 do Watkins\nocite{Watkins:2004:fundamentals}] Escolha convenientemente $\overline{\epsilon}$ para demonstrar a tese para cada $\epsilon > 0$ e manter a ordena\c{c}\~{a}o decrescente dos valores singulares de $A_\epsilon$.
        \begin{solution}
            Seja $A = U D V^t$, onde $U : m \times m$ e $V : n \times n$ s\~{a}o matrizes ortogonais e $D : m \times n$ \'{e} tal que $d_{ij} = 0$ para $i \neq j$, $d_{ii} = \sigma_i$, $i = 1, 2, \ldots, r$, $r < \min\left\{ m, n \right\}$, $\sigma_1 \geq \sigma_2 \geq \ldots \geq \sigma_r > 0$ e $d_{ii} = 0$, $i = r + 1, \ldots, \min\left\{ m, n \right\}$. 
            
            Dado $\epsilon > 0$ construimos $A_\epsilon = U D_\epsilon V^t$ onde $D_\epsilon$ difere de $D$ nas posi\c{c}\~{o}es $d_{ii}$, $i = 1, 2, \ldots, \min\left\{ m, n \right\}$, onde \'{e} acrescido $\bar{\epsilon} < \epsilon$ ao valor de $d_{ii}$.

            Ent\~{a}o, temos que
            \begin{align*}
                \| A - A_\epsilon \|_2 &= \| V^t D U - V^t D_\epsilon U \|_2 \\
                &= \| V^t \left( D - D_\epsilon \right) U \|_2 \\
                &= \| V^t \tilde{D} U \|_2,
            \end{align*}
            onde $\tilde{D}$ \'{e} tal que $d_{ij} = 0$ para $i \neq j$, $d_{ii} = \bar{\epsilon}$, $i = 1, 2, \ldots, \min\left\{ m, n \right\}$.

            Como sabemos que $\| A \|_2 = \sigma_1$ concluimos que $\| A - A_\epsilon \|_2 = \bar{\epsilon} < \epsilon$.
        \end{solution}

        \part A matriz $A_\epsilon$ pode ser constru\'{i}da a partir da SVD de $A$, mantendo as matrizes $U$ e $V$ e adicionando $\overline{\epsilon} < \epsilon$ a cada valor singular de $A$ para compor a matriz $D_\epsilon$?
        \begin{solution}
            Sim, foi esse o m\'{e}todo utilizado no item anterior.
        \end{solution}
    \end{parts}

    \question[Ver p\'{a}gina 271 do Watkins\nocite{Watkins:2004:fundamentals}] Julgue verdadeiro ou falso a afirma\c{c}\~{a}o:
    \begin{quote}
        O conjunto das matrizes de posto completo \'{e} denso e aberto.
    \end{quote}
    Demonstre se a afirma\c{c}\~{a}o for verdadeira e d\^{e} um contra-exemplo se for falsa.
    \begin{solution}
        Em topologia, o conjuntos das matrizes de posto completo \'{e} um subconjunto do $\mathbb{R}^{n \times n}$ aberto e denso, i.e.,
        \begin{enumerate}
            \item Se $A$ tem posto completo, ent\~{a}o todas as matrizes suficientemente pr\'{o}ximas de $A$ tamb\'{e}m tem posto completo. \label{cond1:denso_aberto}
            \item Toda matriz de posto incompleto possue pelo menos uma matriz de posto completo arbitrarirmente pr\'{o}xima dela. \label{cond2:denso_aberto}
        \end{enumerate}
        No exerc\'{i}cio anterior provamos a condi\c{c}\~{a}o \ref{cond2:denso_aberto}. E a demonstra\c{c}\~{a}o da condi\c{c}\~{a}o \ref{cond1:denso_aberto} \'{e} an\'{a}loga a da condi\c{c}\~{a}o \ref{cond2:denso_aberto} mudando apenas o fato da matriz ter posto completo.
    \end{solution}

    \question Demonstre que toda matriz $A : m \times n$ \'{e} limite de uma sequencia de matrizes de posto completo.
    \begin{solution}
        Demonstrar que toda matriz $A : m \times n$ \'{e} limite de uma sequencia de matrizes de posto completo equivale a mostrar que o conjunto das matrizes de posto completo \'{e} denso. No exerc\'{i}cio anterior j\'{a} foi demonstrado que o conjunto das matrizes de posto completo \'{e} denso (e aberto).
    \end{solution}

    \question Demonstre que $A : m \times n$ com posto $s$ e $B : m \times n$ tal que $\| B - A \|_2 < \sigma_s$, ent\~{a}o $\mathrm{posto}(B) \geq s$.
    \begin{solution}
        Seja $A = U D V^t$, onde $U : m \times m$ e $V : n \times n$ s\~{a}o matrizes ortogonais e $D : m \times n$ \'{e} tal que $d_{ij} = 0$ para $i \neq j$, $d_{ii} = \sigma_i$, $i = 1, 2, \ldots, s$, $s \leq \min\left\{ m, n \right\}$, $\sigma_1 \geq \sigma_2 \geq \ldots \geq \sigma_s > 0$ e $d_{ii} = 0$, $i = s + 1, \ldots, \min\left\{ m, n \right\}$. 

        Seja $B = U \tilde{D} V^t$, onde $\tilde{D}$ \'{e} tal que $\tilde{d}_{ij} = 0$ para $i \neq j$, $\tilde{d}_{ii} = \tilde{\sigma}_i$, $i = 1, 2, \ldots, r$, $r \leq \min\left\{ m, n \right\}$, $\tilde{\sigma}_1 \geq \tilde{\sigma}_2 \geq \ldots \geq \tilde{\sigma}_r > 0$ e $d_{ii} = 0$, $i = r + 1, \ldots, \min\left\{ m, n \right\}$. 

        Ent\~{a}o, temos que
        \begin{align*}
            \| B - A \|_2 &= \| U \tilde{D} V^t - U D V^t \|_2 \\
            &= \| U \left( \tilde{D} - D \right) V^t \|_2.
        \end{align*}
        Como desejamos que $\| B - A \|_2 < \sigma_s$ temos que $\tilde{\sigma_i} = \sigma_i$ para $i = 1, \ldots, s$, $\tilde{\sigma_i} > 0$ para $i = s + 1, \ldots, r$ e $\tilde{\sigma_i} = 0$ para $i = r + 1, \ldots, \min{m, n}$. Logo, conclui-se que o $\mathrm{posto}(B) \geq s$.
    \end{solution}

    \question Obtenha a decomposi\c{c}\~{a}o SVD de uma matriz de posto $1$, i.e., se $w : m \times 1$ e $z : n \times 1$ s\~{a}o vetores n\~{a}o nulos, obtenha a decomposi\c{c}\~{a}o SVD de $A = w z^t$.
    \begin{solution}
        Calculamos
        \begin{align*}
            A A^t & \left( w z^t \right) \left( w z^t \right)^t \\
            &= w z^t z w^t \\
            &= \| z \|_2^2 w w^t.
        \end{align*}
    \end{solution}

    \question Considere $A : m \times n$, $m > n$, o vetor $z : m \times n$ e a matriz $B = [A \ z]$. Mostre que $\sigma_{n + 1}(B) \leq \sigma_n(A)$ e $\sigma_1(B) \geq \sigma_1(A)$.
    \begin{solution}
        Temos que
        \begin{align*}
            \sigma_1(B) &= \max_{x \neq 0} \frac{\| B x \|_2}{\| x \|_2} \\
            &\geq \max_{x \neq 0, x_{n + 1} = 0} \frac{\| B x \|_2}{\| x \|_2} \\
            &= \max{x \neq 0} \frac{\| A x \|_2}{\| x \|_2} \\
            &= \sigma_1(A).
        \end{align*}
        % TODO Fazer esse exerc\'{i}cio.
    \end{solution}

    \question Considere agora $A : m \times n$, $m \geq n$, o vetor $z : n \times n$ e a matriz $B = [A; z^t]$. Mostre que $\sigma_n(B) \geq \sigma_n(A)$ e $\sigma_1(A) \leq \sigma_1(B) \leq \sqrt{\left( \sigma_1(A) \right)^2 + \| z \|_2^2}$.
    \begin{solution}
        Temos que
        \begin{align*}
            \sigma_1(B) &= \max_{\| x \|_2 = 1} \frac{\| B x \|_2}{\| x \|_2} \\
            &= \max_{\| x \|_2 = 1} \frac{\| [A; z^t] x \|_2}{\| x \|_2}
        \end{align*}
        % TODO Fazer esse exerc\'{i}cio.
    \end{solution}

    \question[Ver exerc\'{i}cio 4.2.10 do Watkins\nocite{Watkins:2004:fundamentals}] Considere $A : n \times n$ sim\'{e}trica definida positiva. Demonstre que existe $M$ sim\'{e}trica, tal que $A = M^2$. Sugest\~{a}o: use a SVD do fator de Cholesky de $A$).
    \begin{solution}
        Como $A$ \'{e} sim\'{e}trica definida positiva existe $G$ tal que $A = G G^t$. Seja $G = U D V^t$, ent\~{a}o
        \begin{align*}
            A &= G^t G \\
            &= \left( U D V^t \right)^t \left( U D V^t \right) \\
            &= V D^t U^t U D V^t \\
            &= V D^t I D V^t \\
            &= V D^t D V^t \\
            &= V D D V^t \\
            &= V D^2 V^t \\
            &= D^2 V V^t \\
            &= D^2.
        \end{align*}
    \end{solution}

    \question Valores singulares extremos do produto entre duas matrizes. Se $A : m \times p$ e $B : p \times n$, $q = \min\left\{ m, n \right\}$ e $S = \min\left\{ p, n \right\}$, mostre que
    \begin{parts}
        \part $\sigma_1(A B) \leq \sigma_1(A) \sigma_1(B)$,
        \begin{solution}
            Temos que
            \begin{align*}
                \sigma_1(A B) &= \max_{\| x \| = 1} \frac{\| A B x \|_2}{\| x \|_2} \\
                &\leq \max_{\| x \| = 1} \frac{\| A \|_2 \| B x \|_2}{\| x \|_2} \\
                &= \| A \|_2 \max_{\| x \| = 1} \frac{\| B x \|_2}{\| x \|_2} \\
                &= \| A \|_2 \| B \|_2 \\
                &= \sigma_1(A) \sigma_1(B).
            \end{align*}
        \end{solution}

        \part $\sigma_q(A B) \leq \sigma_1(A) \sigma_s(B)$.
        \begin{solution}
            Temos tamb\'{e}m que
            \begin{align*}
                \sigma_q(A B) &= \min_{x \neq 0} \frac{\| A B x \|_2}{\| x \|_2} \\
                &\leq \min_{x \neq 0} \frac{\| A \|_2 \| B x \|_2}{\| x \|_2} \\
                &= \| A \|_2 \min_{x \neq 0} \frac{\| B x \|_2}{\| x \|_2} \\
                &= \sigma_1(A) \sigma_s(B).
            \end{align*}
        \end{solution}
    \end{parts}

    \question[Exerc\'{i}cio 4.1.17 do Watkins\nocite{Watkins:2004:fundamentals}] Demonstre que uma matriz $A : m \times n$ pode ser expressa como $A = U D V^t$ onde $U : m \times m$ e $V : n \times n$ s\~{a}o ortogonais e $D : m \times n$ \'{e} matriz ``diagonal'' com elementos na diagonal $\sigma_1 \geq \sigma_2 \geq \ldots \geq \sigma_r \geq 0$.
    \begin{solution}
        Vamos provar por indu\c{c}\~{a}o finita em $r$, o posto da matriz $A$.
        \begin{enumerate}
            \item Suponha que $A \in \mathbb{R}^{n \times n}$ tem posto $1$. Seja $u_1 \in \mathbb{R}^n$ um vetor em $\mathcal{I}(A)$ tal que $\| u_1 \|_2 = 1$. Ent\~{a}o, como $u_1 \in \mathcal{I}(A)$ verifica-se que existe $x$ tal que
                \begin{align*}
                    A_{\mdot 1} x_1 + A_{\mdot 2} x_2 + \ldots A_{\mdot n} x_n &= u_1
                \end{align*}
                e como o $\mathrm{posto}(A) = 1$ existe apenas uma coluna linearmente independente em $A$ de modo que reescrevemos a igualdade acima como
                \begin{align*}
                    A_{\mdot 1} \left( c_1 x_1 + c_2 x_2 + \ldots + c_n x_n \right) &= u_1,
                \end{align*}
                onde $c_1, c_2, \ldots, c_n \in \mathbb{R}$. Logo, verifica-se que cada coluna de $A$ \'{e} um multiplo de $u_1$.

                Como $A$ \'{e} uma matriz de posto $1$ ela pode ser escrita na forma $A = \sigma_1 u_1 v_1^t$, onde $v_1 \in \mathbb{R}^m$, $\| v_1 \|_2 = 1$ e $\sigma_1 > 0$.

            \item Agora demonstremos que a existe uma matriz ortogonal $U \in \mathbb{R}^{n \times n}$ cuja primeira coluna \'{e} $u_1$.

                De maneira similar, existe uma matriz ortogonal $V \in \mathbb{R}^{m \times m}$ cuja primeira coluna \'{e} $v_1$.

                E, por fim, que $A = U D V^t$, onde $D \in \mathbb{R}^{n \times m}$ possui um \'{u}nico elemento n\~{a}o zero, $\sigma_1$, na posi\c{c}\~{a}o $(1,1)$.

            \item Agora, suponha que $A \in \mathbb{R}^{n \times m}$ possue posto $r > 1$. Seja $v_1$ o vetor unit\'{a}rio tal que $\| A v_1 \|_2 = \max_{\| u \|_2 = 1} \| A u \|_2$. Seja $\sigma_1 = \| A u_1 \|_2 = \| A \|_2$ e $\| u_1 = \sigma_1^{-1} A v_1$. Seja $\tilde{U} \in \mathbb{R}^{n \times n}$ e $\tilde{V} \in \mathbb{R}^{m \times m}$ matrizes ortogonais com primeira coluna $u_1$ e $v_1$, respecitivamente. Seja $\tilde{A} = \tilde{U}^t A \tilde{V}$ tal que $A = \tilde{U} \tilde{A} \tilde{V}^t$. Mostremos que $\tilde{A}$ possue a forma
                \begin{align*}
                    \tilde{A} = \begin{bmatrix}
                        \sigma_1 & z^t \\
                        0 & \hat{A}
                    \end{bmatrix},
                \end{align*}
                onde $z \in \mathbb{R}^{m -1}$ e $\hat{A} \in \mathbb{R}^{(n - 1) \times (m - 1)}$.

            \item Mostre

            \item Most
        \end{enumerate}
        % TODO Fazer esse exerc\'{i}cio.
    \end{solution}
\end{questions}
\bibliographystyle{plain}
\bibliography{bibliography}
\end{document}
