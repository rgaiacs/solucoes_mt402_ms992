% Filename: lista10.tex
% This code is part of 'Listas de Matrizes'
% 
% Description: Lista 10.
% 
% Created: 24.04.12 05:54:57 PM
% Last Change: 25.04.12 09:18:59 AM
% 
% Author: <+AUTHOR+>, <+EMAIL+>
% Organization: <+ORGANIZATION+>
% 
% Copyright (c) <+YEAR+>, <+AUTHOR+>. All rights reserved.
% 
% This file is license under the terms of 
%
\documentclass[a4paper,12pt, leqno, answers]{exam}
\usepackage[top=3cm, bottom=3cm, left=2cm, right=2cm]{geometry}
\usepackage[utf8]{inputenc}
\usepackage[brazil]{babel}
\usepackage{amsmath}
\usepackage{amsfonts}
\usepackage{hyperref}

% Customiza\c{c}\~{a}o da classe exam
\firstpageheader{MT402}{Solu\c{c}\~{a}o da Lista 10}{1º semestre de 2012}
\firstpageheadrule
\footer{Dispon\'{i}vel em \\\url{https://github.com/r-gaia-cs/solucoes_lista_matrizes}
}{}{Reportar erros para \\% Filename: maintainer.tex
% 
% This code is part of 'Solutions for MT402, Matrizes'
% 
% Description: This file keeps the email of the mainteiner.
% 
% Created: 08.05.12 09:46:55 PM
% Last Change: 04.06.12 10:42:01 PM
% 
% Authors:
% - Raniere Silva (2012): initial version
% 
% Copyright (c) 2012 Raniere Silva <r.gaia.cs@gmail.com>
% 
% This work is licensed under the Creative Commons Attribution-ShareAlike 3.0 Unported License. To view a copy of this license, visit http://creativecommons.org/licenses/by-sa/3.0/ or send a letter to Creative Commons, 444 Castro Street, Suite 900, Mountain View, California, 94041, USA.
%
% This work is distributed in the hope that it will be useful, but WITHOUT ANY WARRANTY; without even the implied warranty of MERCHANTABILITY or FITNESS FOR A PARTICULAR PURPOSE.
%
\href{mailto:r.gaia.cs@gmail.com}{r.gaia.cs@gmail.com}
}
\footrule 
\pagestyle{foot}
\renewcommand{\solutiontitle}{\noindent\textbf{Solu\c{c}\~{a}o:}\enspace}
\SolutionEmphasis{\itshape}
\unframedsolutions
\pointname{}

% Customiza\c{c}\~{a}o do pacote amsmath
\allowdisplaybreaks[4]

%Novos ambientes
\newenvironment{fwsolution}{\begin{EnvFullwidth}\begin{TheSolution}}{\end{TheSolution}\end{EnvFullwidth}}

% Novos comandos
%\newcommand{\mdot}{\text{\LARGE $\boldsymbol{\cdot}$}}
\newcommand{\mdot}{\bullet}

\begin{document}
\thispagestyle{headandfoot}
\begin{questions}
    \question Considere $A : m \times n$, $m \geq n$ e $\bar{x} \in \mathbb{R}^n$. Demonstre que: $\| b - A \bar{x} \|_2 = \min_{y \in \mathbb{R}^n} \| b - A y \|_2$ se e somente se $(b - A \bar{x}) \in \text{Im}(A)^\perp$.
    \begin{solution}
        Na p\'{a}gina 438 do Meyer\nocite{Meyer:2000:matrix}, temos que o problema de m\'{i}nimos quadrados consiste em encontrar $x$ tal que $A x = P_{\text{Im}(A)} b$. Mas este sistema \'{e} tal que
        \begin{align*}
            A x = P_{\text{Im}(A)} b &\leftrightarrow P_{\text{Im}(R)} A x = P_{\text{Im}(A)} b \\
            &\leftrightarrow P_{\text{Im}(A)} \left( A x - b \right) = 0 \\
            &\leftrightarrow \left( A x - b \right) \in \text{N}(P_{\text{Im}(A)}) = \text{Im}(A)^\perp = \text{N}(A^t).
        \end{align*}
    \end{solution}
    \question Seja $\bar{x} \in \mathbb{R}^n$. Ent\~{a}o, $\bar{x}$ resolve o problema de quadrados m\'{i}nimos para $A x = b$, se e somente se $A^t A \bar{x} = A^t b$.
    \begin{solution}
        Se $\bar{x}$ \'{e} a solu\c{c}\~{a}o do problema de quadrados m\'{i}nimos de $A x = b$ ent\~{a}o
        \begin{align*}
            A \bar{x} = b &\leftrightarrow Q^t A \bar{x} = Q^t b \\
            &\leftrightarrow R \bar{x} = Q^t b \\
            &\leftrightarrow R^t R \bar{x} = R^t Q^t b \\
            &\leftrightarrow R^t Q^t Q R \bar{x} = R^t Q^t b \\
            &\leftrightarrow A^t A \bar{x} = A^t.
        \end{align*}
    \end{solution}

    \question Considere $A : m \times n$, $m \geq n$ e sua fatora\c{c}\~{a}o ortogonal dada por $A = Q R$ com $Q : m \times m$ ortogonal e $R : m \times n$, com coeficientes $r_{ij} = 0$, $i > j$. Reescreva a solu\c{c}\~{a}o de quadrados m\'{i}nimos $\bar{x} = \left( A^t A \right)^{-1} A^t b$ e a proje\c{c}\~{a}o ortogonal de $b \in \text{Im}(A)$ usando a fatora\c{c}\~{a}o ortogonal de $A$. Relacione as express\~{o}es obtidas com a solu\c{c}\~{a}o de quadrados m\'{i}nimos obtida diretamente da fatora\c{c}\~{a}o ortogonal: $\| b - A x \|_2 = \| Q^t \left( b - A x \right) \|_2$ \ldots
    \begin{solution}
        Podemos reescrever a solu\c{c}\~{a}o da seguinte forma:
        \begin{align*}
            \bar{x} &= \left( A^t A \right)^{-1} A^t b \\
            &= \left( R^t Q^t Q R \right)^{-1} R^t Q^t b \\
            &= \left( R^t R \right)^{-1} R^t Q^t b \\
            &= R^{-1} R^{-t} R^t Q^t b \\
            &= R^{-1} Q^t b.
        \end{align*}

        E a proje\c{c}\~{a}o ortogonal de $b \in \text{Im}(A)$ corresponde a
        \begin{align*}
            P_{\text{Im}(A)} b &= (I - Q) b.
        \end{align*}
    \end{solution}

    \question Considere
    \begin{align*}
        A = \begin{bmatrix}
            1 & 2 \\
            0 & 1 \\
            1 & 0
        \end{bmatrix}.
    \end{align*}
    \begin{parts}
        \part Construa o projetor ortogonal $P$ sobre o subespa\c{c}o $\text{Im}(A)$. Obtenha a proje\c{c}\~{a}o do vetor $v = \left( 1, -1, 2 \right)^t$ em $\text{Im}(A)$. Idem para o vetor $w = \left( 1, 1, -1 \right)^t$.
        \begin{solution}
            Na p\'{a}gina 430 do Meyer\nocite{Meyer:2000:matrix}, temos:
            \begin{quote}
                Seja $\mathcal{M}$ um subespa\c{c}o $r$-dimensional de $\mathbb{R}$, e seja as colunas de $M_{n \times r}$ e $N_{n \times {n - r}}$ as bases de $\mathcal{M}$ e $\mathcal{M}^\perp$, respectivamente. Ent\~{a}o os projetores ortogonais de $\mathcal{M}$ e $\mathcal{M}^\perp$ s\~{a}o
                \begin{align*}
                    P_{\mathcal{M}} &= M \left( M^t M \right)^{-1} M^t \text{ e } P_{\mathcal{M}^\perp} = N \left( N^t N \right)^{-1} N^t.
                \end{align*}
                Se $M$ e $N$ s\~{a}o bases ortonormais para $\mathcal{M}$ e $\mathcal{M}^\perp$, ent\~{a}o
                \begin{align*}
                    P_{\mathcal{M}} &= M M^t \text{ e } P_{\mathcal{M}^\perp} N N^t.
                \end{align*}
            \end{quote}

            Ent\~{a}o,
            \begin{align*}
                P = A \left(  \right)<++>
            \end{align*}<++>
        \end{solution}<++>
    \end{parts}<++>
\end{questions}<++>
\bibliographystyle{plain}
\bibliography{bibliography}
\end{document}
