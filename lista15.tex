% Filename: lista15.tex
% 
% This code is part of 'Solutions for MT402, Matrizes'
% 
% Description: This file corresponds to the solutions of homework sheet 15.
% 
% Created: 08.05.12 09:46:55 PM
% Last Change: 29.06.12 05:41:29 PM
% 
% Authors:
% - Raniere Silva (2012): initial version
% 
% Copyright (c) 2012 Raniere Silva <r.gaia.cs@gmail.com>
% 
% This work is licensed under the Creative Commons Attribution-ShareAlike 3.0 Unported License. To view a copy of this license, visit http://creativecommons.org/licenses/by-sa/3.0/ or send a letter to Creative Commons, 444 Castro Street, Suite 900, Mountain View, California, 94041, USA.
%
% This work is distributed in the hope that it will be useful, but WITHOUT ANY WARRANTY; without even the implied warranty of MERCHANTABILITY or FITNESS FOR A PARTICULAR PURPOSE.
%
Para $\lambda \in \sigma(A) = \left\{ \lambda_1, \lambda_2, \ldots, \lambda_s \right\}$, adotaremos as seguintes defini\c{c}\~{o}es:
\begin{itemize}
    \item A multiplicidade alg\'{e}brica de $\lambda$, denotado por $\mathrm{multAlg}_A(\lambda)$, \'{e} o n\'{u}mero de vezes que $\lambda$ \'{e} raiz do polin\^{o}mio caracter\'{i}stico de $A$.
    \item A multiplicidade geom\'{e}trica de $\lambda$, denotado por $\mathrm{mulGeo}_A(\lambda)$, \'{e} $\mathrm{dim}(\mathcal{N}(A - \lambda I))$, i.e., o n\'{u}mero m\'{a}ximo de autovetores linearmente independetes associados com $\lambda$.
\end{itemize}
\begin{questions}
    \question Sejam $A$ e $B$ matrizes semelhantes (existe $S$, matriz n\~{a}o singular tal que $B = S^{-1} A S)$. Prove que:
    \begin{parts}
        \part $A$ e $B$ t\^{e}m os mesmos autovalores com mesma multiplicidade alg\'{e}brica e geom\'{e}trica.
        \begin{solution}
            Primeiro vamos mostrar que os autovalores de $A$ e $B$ possuem mesma multiplicidade alg\'{e}brica.
            \begin{align*}
                \det(B - \lambda I) &= \det(S^{-1} A S - \lambda I) \\
                &= \det(S^{-1} A S - \lambda S^{-1} S) \\
                &= \det(S^{-1} (A - \lambda I) S) \\
                &= \det(S^{-1}) \det(A - \lambda I) \det(S) \\
                &= \det(A - \lambda I).
            \end{align*}

            Agora vamos mostrar que os autovalores de $A$ e $B$ possuem mesma multiplicidade geom\'{e}trica. Suponha que $(\lambda, x)$ \'{e} um autopar de $B$, ent\~{a}o
            \begin{align*}
                B x &= \lambda x \\
                S^{-1} A S x &= \lambda x \\
                A (S x) &= \lambda (S x).
            \end{align*}
            Como $S$ \'{e} n\~{a}o-singular ent\~{a}o todo autovetor de $B$ est\'{a} mapeado uma \'{u}nica vez em $A$ e portanto $A$ e $B$ possuem a mesma multiplicidade geom\'{e}trica.
        \end{solution}
        
        \part Se $(\lambda_i, v_i)$ \'{e} autopar de $A$ como \'{e} o autopar $(\lambda_i, w_i)$ de $B$?
        \begin{solution}
            \begin{align*}
                B w_i &= \lambda_i w_i \\
                S^{-1} A S w_i &= \lambda_i w_i \\
                A (S w_i) &= \lambda_i (S w_i).
            \end{align*}
        \end{solution}
    \end{parts}

    \question Considere $A \in \mathbb{R}^{n \times n}$. Demonstre que:
    \begin{parts}
        \part os autovalores complexos de $A$ ocorrem em pares conjugados de mesma multiplicidade alg\'{e}brica.
        \begin{solution}
            Seja $(x + \lambda_1) (x + \lambda_2) \ldots (x + \lambda_n) = 0$ o polin\^{o}mio caracter\'{i}stico de $A$. Como $A \in \mathbb{R}^{n \times n}$ temos $\lambda_1 \lambda_2 \ldots \lambda_n \in \mathbb{R}$ e, portanto, se existe um $\lambda_i \in \mathbb{C}$ ent\~{a}o existe um $\lambda_j = \overline{\lambda_i}$, $j \neq i$, e com mesma multiplicidade.
        \end{solution}
        \part se $\lambda$ \'{e} complexo e se $v$ \'{e} seu autovetor associado ent\~{a}o $\bar{v}$ \'{e} autovetor associado \`{a} $\bar{\lambda}$ e, se representarmos $v$ por $v = a + b i$, com $a, b \in \mathbb{R}^n$ ent\~{a}o, $a$ e $b$ s\~{a}o linearmente independentes.
        \begin{solution}
            Se $(\lambda, v)$, $\lambda \in \mathbb{C}$, \'{e} um autopar ent\~{a}o $A v = \lambda v$. Logo,
            \begin{align*}
                \overline{A v} &= \overline{\lambda v} \\
                \overline{A} \overline{v} &= \overline{\lambda} \overline{v} \\
                A \overline{v} &= \overline{\lambda} \overline{v}.
            \end{align*}

            Se representarmos $v$ por $v = a + b i$, com $a, b \in \mathbb{R}^n$ ent\~{a}o $a$ e $b$ s\~{a}o linearmente independentes pois se estes forem linearmente dependentes ent\~{a}o $a$ e $b$ seriam autovetores.
        \end{solution}
    \end{parts}

    \question[Equa\c{c}\~{a}o 7.2.2, p\'{a}gina 511, do Meyer\nocite{Meyer:2000:matrix}] Demonstre que para cada matriz $A \in \mathbb{C}^{n \times n}$ e cada $\lambda \in \sigma(A)$: $\mathrm{multGeo}_A(\lambda) \leq \mathrm{multAlg}_A(\lambda)$.
    \begin{solution}
        Suponha que $\mathrm{multAlg}_A(\lambda) = k$. O teorema da triangulariza\c{c}\~{a}o de Schur garante que existe uma matriz $U$ unit\'{a}ria tal que
        \begin{align*}
            U^* A U &= \begin{bmatrix}
                T_{11} & T_{12} \\
                0 & T_{22}
            \end{bmatrix},
        \end{align*}
        onde $T_{11} \in \mathbb{C}^{k \times k}$ \'{e} uma matrix triangular superior cuja entradas da diagonal s\~{a}o iguais a $\lambda$ e $T_{22} \in \mathbb{C}^{(n - k) \times (n - k)}$ \'{e} uma matriz triangular superior tal que $\lambda \not\in \sigma(T_{22})$. Consequentemente, $T_{22} - \lambda I$ \'{e} n\~{a}o singular e portanto
        \begin{align*}
            \mathrm{posto}(A - \lambda I) &= \mathrm{posto}\left( U^* (A - \lambda I) U \right) \\
            &= \mathrm{posto}\begin{bmatrix}
                T_{11} - \lambda I & T_{12} \\
                0 & T_{22} - \lambda I
            \end{bmatrix} \\
            &\geq \mathrm{posto}(T_{22} - \lambda I) \\
            &= n - k,
        \end{align*}
        onde a desigualdade seque do fato que o posto de uma matriz \'{e} no m\'{i}nimo t\~{a}o grande quanto o posto de qualquer submatriz. Logo
        \begin{align*}
            \mathrm{multAlg}_A(\lambda) &= k \\
            &\geq n - \mathrm{posto}(A - \lambda I) \\
            &= \dim(\mathcal{N}(A - \lambda I)) \\
            &= \mathrm{multGeo}_A(\lambda).
        \end{align*}
    \end{solution}

    \question[Equa\c{c}\~{a}o 7.2.5, p\'{a}gina 512, do Meyer\nocite{Meyer:2000:matrix}] Demonstre que $A$ \'{e} diagonaliz\'{a}vel se e somente se $\mathrm{multGeo}_A(\lambda) = \mathrm{multAlg}_A(\lambda)$ para cada $\lambda \in \sigma(\lambda)$.
    \begin{solution}
        Suponha que $\mathrm{multGeo}_A(\lambda_i) = \mathrm{multAlg}_A(\lambda_i) = a_i$ para todo autovalor $\lambda_i$. Se existem $k$ autovalores distintos e se $B_i$ \'{e} uma base para $\mathcal{N}(A - \lambda_i I)$, ent\~{a}o $B = B_1 \cup B_2 \cup \ldots \cup B_k$ cont\'{e}m $\sum_{i = 1}^k a_i = n$ vetores. Como $B$ \'{e} um conjunto linearmente independente, ent\~{a}o $B$ representa o conjunto completo de autovetores linearmente independentes de $A$, e portanto $A$ deve ser diagonaliz\'{a}vel.

        Se $A$ \'{e} diagonaliz\'{a}vel e se $\lambda$ \'{e} um autovalor de $A$ com $\mathrm{multAlg}_A(\lambda) = a$ ent\~{a}o existe uma matriz n\~{a}o singular $P$ tal que
        \begin{align*}
            P^{-1} A P &= D = \begin{bmatrix}
                \lambda I_{a \times a} & 0 \\
                0 & B
            \end{bmatrix},
        \end{align*}
        onde $\lambda \not\in \sigma(B)$. Consequentemente,
        \begin{align*}
            \mathrm{posto}(A - \lambda I) &= \mathrm{posto} P \begin{bmatrix}
                0 & \ 7 \\
                0 & B - \lambda I
            \end{bmatrix} P^{-1} \\
            &= \mathrm{posto}(B - \lambda I) \\
            &= n - a,
        \end{align*}
        e portanto
        \begin{align*}
            \mathrm{multGeo}_A(\lambda) &= \dim(\mathcal{N}(A - \lambda I)) \\
            &= n - \mathrm{posto}(A - \lambda I) \\
            &= a \\
            &= \mathrm{multAlg}_A(\lambda).
        \end{align*}
    \end{solution}

    \question[Equa\c{c}\~{a}o 7.2.6, p\'{a}gina 514, do Meyer\nocite{Meyer:2000:matrix}] Se $A$ possui $n$ autovalores distintos, $A$ \'{e} diagonaliz\'{a}vel? E se $A$ \'{e} diagonaliz\'{a}vel, ent\~{a}o possui $n$ autovalores distintos?
    \begin{solution}
        Se $A_{n \times n}$ possue $n$ autovalores distintos ent\~{a}o $\text{multGeo}_A(\lambda) = \text{multAlg}_A(\lambda) = 1$ para cada $\lambda$ e portanto a matriz $A$ \'{e} diagonaliz\'{a}vel.

        \'{E} poss\'{i}vel que $\text{multGeo}_A(\lambda) = \text{multAlg}_A(\lambda)$, i.e., seja diagonaliz\'{a}vel, e $A$ possua pelo menos um autovalor com multiplicidade maior ou igual a $2$. A matriz abaixo \'{e} um exemplo:
        \begin{align*}
            \begin{bmatrix}
                1 & -4 & -4 \\
                8 & -11 & - 8 \\
                -8 & 8 & -5
            \end{bmatrix}.
        \end{align*}
    \end{solution}

    \question Para $A : n \times n$ e $\sigma(A) = \{ \lambda_1, \ldots, \lambda_n \}$. Demonstre a equival\^{e}ncia entre as afirma\c{c}\~{o}es:
    \begin{enumerate}
        \item $A$ \'{e} semelhante a uma matriz diagonal: $A = P^{-1} D P$;
        \item $A$ possui um conjunto completo de autovetores lineamente independentes;
        \item $\mathrm{multGeo}(\lambda_i) = \mathrm{multAlg}(\lambda_i)$, para $i = 1, 2, \ldots, n$.
    \end{enumerate}
    \begin{solution}
        % TODO Fazer esse exerc\'{i}cio.
    \end{solution}

    \question[Exemplo 7.5.1, p\'{a}gina 549, do Meyer\nocite{Meyer:2000:matrix}] Considere $A \in \mathbb{C}^{n \times n}$ e o vetor $q \in \mathbb{C}^n$. Demonstre que $\overline{\rho} = \left( q^* A q \right) / \left( q^* q \right)$ minimiza $\| A q - \rho q \|_2$.

    Observe que $A q = \rho q$ \'{e} um sistema sobredeterminado e considere que as equa\c{c}\~{o}es normais para um sistema complexo, $C z = b$, s\~{a}o daddas por $C^* C z = C^* b$.
    \begin{solution}
        Sabemos que $\min \| A q - \rho q \|_2$ equivale a resolver o sistema normal dado por
        \begin{align*}
            A^* A q &= A^* \rho q = \rho A^* q.
        \end{align*}<++>
    \end{solution}

    \question Considere $A_{n \times n}$ hermitiana. Demonstre que $\lambda_1 = \max x^* A x$ e $\lambda_n = \min x^* A x$, para $x \in \mathbb{C}^n$, $\| x \|_2 = 1$, e $\lambda_1 \geq \lambda_2 \geq \ldots \geq \lambda_n$. Interprete este resultado fazendo uma rela\c{c}\~{a}o com o quociente de Rayleigh.
    \begin{solution}
        % TODO Fazer esse exerc\'{i}cio.
    \end{solution}

    \question[Equa\c{c}\~{a}o 7.5.5, p\'{a}gina 550, do Meyer\nocite{Meyer:2000:matrix}] Enuncie e demontre o Teorema de Courant-Fischer.
    \begin{solution}
        % TODO Fazer esse exerc\'{i}cio.
    \end{solution}

    \question Considere que a matriz $A$ real, sim\'{e}trica, n\~{a}o singular e $\mathcal{L} < \min_i \lambda_i$, onde $\lambda_i$ s\~{a}o os autovalores de $A$. A matriz $A + \lambda I$ tem fatora\c{c}\~{a}o de Chlolesky? Justifique.
    \begin{solution}
        % TODO Fazer esse exerc\'{i}cio.
    \end{solution}
    
    \question Considere uma matriz $A_{4 \times 4}$, com autovalores iguais a $-9$, $-8$, $3$ e $11$. Usando o m\'{e}todo das pot\^{e}ncias e suas varia\c{c}\~{o}es, para quais autovalores ser\'{a} poss\'{i}vel obter estimativas? Indique qual a forma do m\'{e}todo das pot\^{e}ncias ser\'{a} usada em cada caso.
    \begin{solution}
        % TODO Fazer esse exerc\'{i}cio.
    \end{solution}

    \question Considere $A_{n \times n}$ e seus autovalores $| \lambda_1 | > | \lambda_2 | \leq | \lambda_3 | \ldots \leq | \lambda_n |$. Supondo que o m\'{e}todo das pot\^{e}ncias gera uma sequ\^{e}ncia de vetores $q^{(k)}$ convergente, qual o limite desta sequ\^{e}ncia? Justifique sua resposta.
    \begin{solution}
        % TODO Fazer esse exerc\'{i}cio.
    \end{solution}

    \question Ainda no exerc\'{i}cio anterior, suponha agora que $| \lambda_1 | = | \lambda_2 | > | \lambda_3 | \leq \ldots \leq | \lambda_n |$. Aplicando o m\'{e}todo das pot\^{e}ncias \`{a} matriz $A$ o que podemos afirmar sobre o vetor limite da sequ\^{e}ncia $\left\{ q^{(k)} \right\}$? Justifique sua resposta.
    \begin{solution}
        % TODO Fazer esse exerc\'{i}cio.
    \end{solution}

    \question Descreva o m\'{e}todo das pot\^{e}ncias inverso com shift.
    \begin{solution}
        % TODO Fazer esse exerc\'{i}cio.
    \end{solution}

    \question Considere o m\'{e}todo da Itera\c{c}\~{a}o de Rayleigh com shift. Qual a rela\c{c}\~{a}o deste m\'{e}todo com o das pot\^{e}ncias inverso com shift? Justifique a escolha do shift neste caso.
    \begin{solution}
        % TODO Fazer esse exerc\'{i}cio.
    \end{solution}
\end{questions}
